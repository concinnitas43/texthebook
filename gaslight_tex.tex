    
\documentclass{book}

\begin{document}

\title{\textbf{GASLIGHTING} \\ \vspace{1em}\Large \textsc{How To Drive Your Enemies Crazy}}
\author{Victor Santoro}
\date{}

\maketitle

\begin{center}
    \textit{
        This book is sold for information purposes only. Neither the 
        author nor the publisher will be held accountable for the use or 
        misuse of the information contained in this book. 
    }
\end{center}

\tableofcontents

\chapter*{Introduction}

"Gaslighting” is a systematic array of techniques that de- 
stroys your target's mental equilibrium, self-confidence, and 
self-esteem, and is designed to drive your target nuts. Gaslight- 
ing is not conventional harassment or physical destruction, but 
highly refined and subtle psychological warfare. 
Relatively crude harassing tactics, such as sending him 
Magazine subscriptions and having his utilities turned off, 
make it unequivocally clear to the target that he's under attack 
by someone else. Harassment techniques are now well-known, 
and this has made it easier to construct defenses against them. 
Harassment may also turn the sympathies of the target's friends 
and associates in his favor. 
Gaslighting is far more subtle, because most of the tech- 
niques never clearly point to a malevolent or vengeful other 
party. The hapless target never suspects that things are being 
done to him; he comes to believe he's doing them to himself! 
Gaslighting destroys his self-confidence and makes him doubt 
his competence and sanity, Other tactics are designed to alien- 
ate him from his family members, neighbors, associates, fellow 
employees, and even his employer!Gaslighting 
2 
The term "Gaslighting" comes from the 1944 Hollywood 
movie, Gaslight, starring Charles Boyer, Ingrid Bergman, and 
Joseph Cotten. In the movie, Charles Boyer tries to convince his 
wife that she's going insane by contriving incidents designed to 
make it appear as if she's forgetful, disoriented, and confused. 
As Ingrid Bergman played a Victorian housewife, Boyer's evil 
character had total control over her environment, and was able 
to use the force of his personality to convince Bergman that she 
had inherited bad genes from her mother, who had died insane. 
In that limited Victorian environment, without electricity, 
telephone, radio, television, and E-Mail, the techniques ayail- 
able were very few, Boyer hid some of her possessions, to make 
it appear that Bergman was becoming acutely forgetful. Room 
lighting was by gas, which provided an opportunity for devious 
mind-damaging. Boyer would sneak into the attic, for example, 
to rummage among the goods stored in it, and turn on the gas- 
light, which would cause the pressure to drop in the bedroom 
below, Ingrid Bergman became so psyched-out that she thought 
she was imagining the dimming of the light in her room. 
Today, it’s possible to employ a variety of methods to de- 
stroy someone's self-confidence and emotional equilibrium. 
Everyday life is so complicated that it provides many avenues 
of attack. We can take advantage of modern technology to 
strike at the target simultaneously from several directions. We 
can convince the target that he keeps losing things, and that his 
memory is slipping disastrously. Gaslighting also destroys the 
target’s credibility with his relatives, associates, and fellow 
employees. We can seriously impair his vocational and social 
relationships, alienating him from his family and friends. Gas- 
lighting serves as a tool of attack, as well as defense against 
other people's unethical behavior. 
Some reckless people have gaslighted their rivals and 
enemies by putting a psychedelic drug such as LSD or mesca- 
  Introduction 
3 
line in their food or drink. Unfortunately, drugs show up on 
blood tests, a very good reason for avoiding their use. These 
extremes, which can get you in serious trouble with drug en- 
forcement officers if you try to buy or administer illegal drugs, 
are unnecessary. As we'll explain, you can gaslight someone 
with commonly available and perfectly legal materials. 
Most of the gaslighting tactics explained here cost nothing 
to execute. A few require a modest investment, such as a 
magazine subscription, a bouquet of flowers, or the price of a 
few keys. None are overly expensive, so gaslighting is within 
reach of even the most modest budgets. 
Keep in mind that not all gaslighting tactics described 
herein are suitable for all targets. Some simply won't work be- 
cause your target doesn't fit the situation, Others will be inap- 
propriate because they don't provide the exact effect you wish, 
Yet others will be out of the question because you lack the right 
resources, Still others will be wrong because they can harm in- 
nocent persons. You have to be inventive, and able to design 
the best blueprint for gaslighting your target. 
Your mental alertness and ingenuity will be your most valu- 
able weapons, because it's impossible to lay out every possible 
variant of every possible tactic in this book. At times, you'll 
have to improvise, adapting existing techniques to your special 
needs, Given the nature of human ingenuity, you'll probably in- 
vent a few new tactics, as well, 
In this book, we'll use the term "gaslight" as a verb. The 
past tense will be "gaslighted," not "gaslit." Someone who 
practices gaslighting will be a "gaslighter,” 
This book will use the masculine pronoun almost exclu- 
sively, except in cases where a female is indicated, although 
obviously most gaslighting tactics are workable against both 
males and females, This may be “sexist” language, but it makes  Gaslighting 
4 
writing and reading easier when we don't struggle through 
awkward phrasing such as "he/she," "his and hers," etc. 
Read and enjoy this, but take it seriously. Even if you have 
no target in mind right now, someone else might be planning to 
gaslight you! 
  
  Chapter One 
Gaslighting Philosophy 
45 

\chapter{Gaslighting Philosophy}
Chapter One 
GASLIGHTING 
PHILOSOPHY 
Gaslighting is a method of attack. The purpose of gaslight- 
ing isn't pure physical destruction, but destruction of your tar- 
get's intangible assets: his confidence, self-esteem, and reputa- 
tion. You accomplish this by delivering a series of mental hot 
foots to destroy his mental equilibrium, something also known 
as "mind-fucking.” Some forceful gaslighting tactics will seri- 
ously disrupt your target's life, but leave him wondering what 
he's been doing to himself. If you plan your attack well, your 
target will soon become tense and anxious, wondering if he's 
losing his grip on reality, and certainly losing sleep over the 
strange experiences he's been having. His family, associates, 
and fellow employees will begin wondering if your target's go- 
ing off the deep end, With just a little bit of luck, you can 
eventually reduce your target to a shapeless mass of shivering, 
quivering jelly, 
Selecting Your Target 
Gaslighting is a powerful method of reprisal, and to make it 
worthwhile, you must be very selective in its application. Gas-    Gaslighting 
6 
lighting isn't a weapon to use against a clumsy or rude waiter, 
or an inconsiderate person who doesn't return your phone call. 
Don't waste your time trying to gaslight someone who flips you 
the bird in traffic, either. Frankly, if you take offense at such 
minor slights, you'll spend your life alienated from most people, 
because people aren't perfect, and sometimes give offense with- 
out malice. In other words, don’t go around with a chip on your 
shoulder and don't be an "injustice collector." Save your 
strength for the significant issues. 
Tactically, it's best to save your efforts for a really important 
target instead of dissipating them on minor annoyances. You'll 
be better off focusing your efforts on a target that really matters 
instead of trying to settle scores with everyone who rubs you 
the wrong way. 
Gaslighting is for people who intentionally hurt you, with 
premeditation, and without consideration for your feelings or 
any attempt to make it up to you. The utterly immoral back- 
stabber is @ prime target. An extreme example is the "best 
friend" who is porking your wife. Another is the fellow em- 
ployee who tells your supervisor malicious lies about you be- 
hind your back. You may also feel inclined toward revenge 
against the bad-mannered and noisy neighbor who persistently 
plays his stereo late at night, and dismisses your requests to 
lower the volume. 
Let The Ponishiment Fit The Crime 
Gaslighting allows you to tailor your tactics to give your 
target the same level of mental anguish he's caused you. This is 
an extremely important point, because the main reason for gas- 
lighting an adversary is to make you feel better. 
We've learned a lot about victims and victimization during 
the Twentieth Century. Police officers quickly learn how vic- 
Chapter One 
Gaslighting Philosophy 
7 
tims react to being victims of crime. Victims often feel violated 
and powerless, and this is tue whether the individual has suf- 
fered criminal victimization or been victimized in a way not 
punishable by law. 
The worst aspect of victimization experienced is the fact 
that aggressors usually get away with it. The rate of clearance 
for burglary, for example, is about 13 percent nationwide. If 
you ever suffer burglary, the odds are overwhelming that the 
burglar will get away with it, and that you'll never see the stolen 
property again. 
Although nobody keeps statistics on non-criminal yictimi- 
zation, it's painfully obvious that ruthless aggressors get away 
with it practically all of the time. The employer, for example, 
who blames you for one of his blunders and fires you for it, is 
almost invulnerable to reprisal. You can, in theory, take him to 
court, but this is a long and rocky road even if you win, and 
"justice delayed is justice denied,” 
Don't Be A Loser 
The victim, aware of his powerlessness, often feels like a 
loser, and he is a loser unless he takes steps to inflict as much 
distress on the aggressor as he's suffered. Sometimes, the urge 
to take reprisal explodes in irrational violence. Watching the 
news, we hear about former employees shooting up post offices, 
restaurants, or workplaces because the stress of unfair treatment 
was too much for them, and they snapped. Unfortunately, such 
actions are clearly illegal, and the former victim loses a second 
time because he's prosecuted if he survives. Some commit sui- 
cide, unwilling to face prosecution. Their relatives suffer, as 
well, because the gunman is labeled as a "mad dog” and the 
person who caused the original problem is treated like a hero or 
victim,    Gaslighting 
8 
If you've become’a victim, the problem is to find a balanced 
‘solution. Doing nothing will leave you feeling helpless and even 
depressed, Over-reacting can cause more trouble for you than 
it's worth. Gaslighting is the happy medium, because it’s flex- 
ible, allowing you to adapt your tactics to each individual 
situation. 
Many crude harassment tactics are obsolete. Many periodi- 
cals demand payment with the subscription, to forestall spuri- 
ous subscriptions. Ordering pizzas, flowers, or liquor for your 
target by telephone is much more difficult today, because sup- 
pliers ask for your telephone number and, in locales with 
"Caller 1.D.," can check right away to ensure that you've given 
the correct number. 
Unlike harassment, gaslighting works best when you're 
close to your target. It'll become obvious, as we study gaslight- 
ing techniques, that it's easier to devastate a close associate than 
a total stranger. 
Progressive Planning 
A fundamental aspect of gaslighting doctrine is that it's a progressive attack upon your target's personality first, followed by his social and occupational relationships. Unlike harass- ment, where you work from the outside in, gaslighting begins at the center and works out. The sequence of tactics is important, because if you were to mount a direct attack first, you'd run into a strong and fully functioning personality, able to resist the as- saults, 
For example, if you were to tamper with your target's e-mail as your first step, he'd be quickly aware that a malevolent influence was at work, and would be on his guard against fur- ther intrusions. Once he thinks that someone's out to get him, he'll be a much harder target to hit effectively. 
Chapter One 
Gaslighting Philosophy 
9 
This is why the first phase in any gaslighting program must 
be to create an aura of self-doubt, and to destroy the target's 
self-confidence. From this, it's easy to understand that gaslight- 
ing is more subtle than harassment, und requires a more delicate 
touch. It's very much like the difference between an injection of 
anesthetic and a blackjack blow to the head. 
Basic Tactics 
There are several steps in preparing a gaslighting campaign 
against a target. The first is to find out all you can about your 
target, to uncover points of entry into his life. The following list 
provides a minimal outline of what you need to know: 
Full name 
Home address 
Home telephone 
Marital status 
Spouse's name 
Number of children, if any 
Children’s sex and age 
Other relatives 
Occupation 
Work address 
Work telephone 
Name of supervisor 
Names of fellow employees 
Spouse's workplace 
Hobbies and interests 
Motor vehicles owned 
License plate numbers 
Credit card accounts 
Bank account numbers SNNNSOW 
SSE 
RSS 
BENE  Gaslighting 
10 
This basic information allows you to study your target's 
lifestyle and plan the best ways to cause him mental anguish 
and dislocation. Beyond the basics, it helps to know the subtle- 
ties of his relationships. How well does he get along with his 
family? His friends? His fellow employees? The more you know 
about him, the more ways you'll find to attack him. Let's ex- 
plore this in detail. 
Exploiting Opportunites 
As important as the basic details is the psychological and 
social picture you find. Is your target normally tense and anx- 
ious? If so, he's already done part of your work for you, as it 
will take less effort to push him over the edge. 
Does he shy away from people? Is he uncomfortable with 
fellow employees and neighbors? If so, look for the reasons, 
which may highlight a plan of attack. 
Is he the company spy, or does he play company politics in 
a particularly cruel and ruthless way? Has he made many ene- 
mies at work, or does he have repeated disputes with his neigh- 
bors? Does he haye many sins on his conscience, aware that 
many people he's hurt are waiting for him to stub his toe, and 
perhaps even give him a little push? It's important to remember 
that the more your target's associates and fellow employees dis- 
like him, the more willing they'll be to believe the worst about 
him. 
The greater the number of people who have it in for your 
target, the easier it will be for you to obtain help for your gas- 
lighting projects. Make a list of his enemies, as well as his well- 
wishers, to provide a framework for recruiting accomplices. 
Why his friends and well-wishers? Because if your target is 
truly a nasty, treacherous person, it's a sure bet that he'll alien- 
ate one or more of his friends with his traitorous tactics, A for- 
Chapter One 
Gaslighting Philosophy 
11 
mer friend, burning over a betrayal, will be eager to exploit the 
opportunity to avenge a stab in the back. As a close associate, 
he'll know some of the details that will open your target up to 
gaslighting. 
Information is the first dimension; access is another. The 
greater your social and physical access to your target, the more 
you'll be able to do to him. This is in sharp contrast to tech- 
niques of harassment, in which you strike from a distance. The 
best situation is if you have both occupational and social con- 
tact with your target. This doesn't mean that he has to be a per- 
sonal friend, and with some notable exceptions, you wouldn't 
be gaslighting a friend. You just need proximity and invisibil- 
ity. If you work for the same company, and attend the same so- 
cial gatherings, both serve as entries into your target's life. As a 
neighbor, you can observe your target and form a detailed pic- 
ture of his schedule, interests, and lifestyle. These points of en- 
try serve both for information and for action against him. 
Let's look at the prospect of invisibility. If you work for the 
same company, you don't even have to be his peer. You can be 
a secretary or even the janitor, because to many people, janitors 
are invisible, We take them for granted, barely acknowledging 
their presence as they go about their work. Your target, evil and 
snotty person he is, will consider the janitor to be beneath his 
notice, and perhaps even beneath his contempt. Likewise, many 
of these unsavory persons consider a secretary merely as one 
who types letters, answers the telephone, and fetches the 
moming coffee. 
As janitor, you may not be able to mix socially with your 
target, but one compensation Is that you have legitimate reason 
for being in the workplace after closing time, and legitimate 
reason for collecting the contents of wastebaskets, which can be 
very revealing. Credit slips, canceled checks, notes and letters,Gaslighting 
12 
envelopes, and other artifacts retrieved from the wastebasket 
can be gold nuggets of information. 
Harassment is as subtle as a firing squad. Gaslighting's 
main element is stealth. Strike from the shadows, with a deli- 
cate hand, to keep the target unaware that he's the subject of a 
thorough campaign of disorientation and dislocation. 
Psychological Pearl Harbor 
Surprise is an essential element in planning an attack, 
whether a military knock-out blow or a psychological assault. 
Your target should not discover that he's under attack until your 
plan is well under way and it's too late for him to counter it. 
This is why you should never display open hostility toward 
him. Revealing your grudge negates the advantage of stealth 
because it alerts your target to the prospect of hostile action 
from you, and makes you the prime suspect when things begin 
happening. It also helps greatly to allow time between the inci- 
dent that provoked you to plan retaliation and the beginning of 
your efforts. This allows the incident to fade from his memory, 
decreasing the chances of his suspecting that you're behind his 
misfortunes. With some gaslighting tactics, allowing time to 
elapse is essential for success, as we'll see. 
Best of all is to avoid revealing that you're even aware that 
your target has done you dirty. Knowledge is power, and keep- 
ing it to yourself multiplies its effectiveness. If, for example, 
you discover your wife cheating with your best friend, don't 
have a confrontation. Pretend to be fat, dumb, and happy while 
you plan your revenge. 
An important tactical point is that you should take advan- 
tage of any social, occupational, or emotional proximity to fur- 
ther your plans. The closer you are to your target, the easier it 
will be to deliver a stab in the back. 
Chapter One 
Gaslighting Philosophy 
13 
Some gaslighting tactics are possible only when you pretend 
to be a close friend or confidant. Intellectual and emotional 
proximity allow you to get inside your target's head and inflict 
serious damage. In some cases, closeness is devastating. 
Employing Accomplices 
While many gaslighting techniques are designed so that you 
can use them alone, others work better with two people. In 
many cases, you'll improve your chances of success if you have 
a confederate or two to support you. This is especially true if 
you're conducting a whispering campaign against your target. A 
target may accuse one person of “having it in for him," but if he 
begins voicing suspicions of several people conspiring against 
him, he'll present an image of paranoia. 
Plan your strategy to make fullest use of the "multiplier ef- 
fect," when one event provokes a series of subsequent events 
that make your target's situation even worse without your hay- 
ing to contribute anything more. Provoking a confrontation 
between your target and his employer is a good example of the 
multiplier effect, because the personal interplay during a con- 
frontation adds gasoline to the flames. We'll study many exam- 
ples of the multiplier effect in the following chapters. 
Make Your Own Luck 
Life is full of surprises, which is why you should always 
remain alert to new opportunities. When you begin a gaslight- 
ing campaign, you won't be able to plan it down to the last dot- 
ted "i" and crossed "t". You may find unexpected opportunities 
presenting themselves, and you must be prepared to jump in 
and take advantage of them before they disappear, An example 
is when a fellow employee is absent. This provides an oppor- 
tunity to search his desk or plant material in his workplace.  Gaslighting 
14 
Sometimes, you can make your own luck by taking advantage 
of a fleeting opportunity, One tactic is to "borrow" his key-ring, 
if he leaves it within your reach, long enough to have duplicates 
made of his keys. These will be extremely valuable when you 
begin gaslighting him. 
The tactics in this book are under various headings, depend- 
ing on whether they produce disorientation, or provoke 
confrontations with family, friends, or fellow employees. Others 
lead your target's friends and associates to begin wondering if 
he's having a mental lapse. However, there's no way to place 
them all in neat pigeonholes, because tactics and effects over- 
lap. Many will have more than one effect, causing the target to 
doubt his sanity while provoking an unwanted confrontation 
with an associate or relative. 
This is why gaslighting tactics are so effective. Often, one 
tactic will have several side-effects, thereby enhancing its 
power. Learn to make the most of this for maximum disruption 
against your target. Remember that there is no defense against a 
well-planned and executed gaslighting attack. 
Practical Preparations 
If you think you may conduct a gaslighting campaign 
against anyone in the future, begin preparing for it now, Some 
practical preparations can facilitate any gaslighting campaign, 
no matter who the target may be, and these are worth doing well 
in advance to avoid having to scramble at the last minute. 
Official stationery from government agencies and 
letterheads from several companies can serve various purposes, 
so whenever you have an opportunity, take a few letterheads 
and envelopes from the office of every company for which you 
work. If you visit other companies, or a government agency, 
and have the opportunity to take a few forms and documents 
Chapter One 
Gastighting Philosophy 
15 
from a secretary's desk or drawer, add them to your file. As 
we'll see, commonplace forms such as hunting licenses and 
vehicle registration forms enable you to use devastating tactics 
against your target. 
Don't forget employment applications. These-also have their 
uses, as we'll see. 
Collect bars of soap and hotel stationery from several hotels 
and motels around the country if you travel, If not, ask a fnend 
who travels to pick up a few for you, explaining that you're a 
collector, You'll have good uses for these later. 
Locate several second-hand thrift stores in your area. These 
will be handy for purchasing used clothing that you can use for 
several gaslighting tactics. 
A couple of pairs of latex or thin plastic gloves may be es- 
sential, if you intend to intercept your target's mail or enter his 
home. It's very unlikely that your fingerprints will pose any 
problems, but it's so simple to avoid leaving any that the pre- 
caution is worth taking. 
Two more preparations involve items that may be illegal, so 
be extremely careful if you decide to obtain them. The first is a 
small amount of an illegal drug, which you'll use only for 
planting in your target's clothing. Never make an illegal drug 
deal yourself, because you might get busted, but if a friend of 
yours uses an illegal substance, scrounge a small amount and 
save it in a plastic envelope for future use. It's best never to 
store illegal drugs where you live or wark. 
The second item is a handgun. In jurisdictions with strict 
gun control laws, obtaining any sort of firearm can be very dif- 
ficult. In other locales, such as most of the "Sunbelt" states, you 
can buy a handgun at a garage sale, no questions asked. Don't 
worry about the quality of the handgun. Just wipe every surface 
with an oily rag after purchase to obliterate your fingerprints. 
Wipe off any cartridges you load into it, then put it in a plastic  Gaslighting 
16 
bag. Store it in a safe place away from home, if handgun own- 
ership is restricted in your area, You'll have several opportuni- 
ties for its use later. 
Storing contraband items requires forethought and caution. 
Don't do anything stupid, such as keeping it in a safe deposit 
box, because a box is traceable back to you. One way is to bury 
it away from your home and workplace. Another is to put it un- 
der a floorboard or behind a loose brick in an abandoned 
building, keeping in mind that someone might accidentally un- 
cover it. Still, it's better to lose your stash than suffer arrest and 
prosecution for its possession. 
If you need to begin gaslighting someone right now, you'll 
have to cut comers. Look over the business letters you've re- 
ceived recently, as you can use some of these for forgeries. You 
type a letter of your own, and put it just below the heading, then 
photocopy the composite. The result will appear to be a copy of 
a genuine letter. 
One last-minute preparation is to order a rubber stamp with 
your target's name and address, This will be useful when you're 
sending "mail" on his behalf, Obviously, pay cash for this, and 
pick it up in person. 
Two Words of Caution 
Some of the tactics laid out here, such as obstructing your 
target's mail, are illegal. This points up the importance of not 
getting caught, although prosecution is unlikely. An important 
corollary to this, worth repeating, is to avoid advertising your 
animosity towards your target, so as not to let him or anyone 
else know that you're carrying a grudge, A show of hostility can 
kick back at you by making you a prime suspect, so you have to 
learn and practice diligent self-control, Gaslighting requires a 
fine hand, to avoid detection both by your target and by the 
Chapter One 
Gaslighting Philosophy 
17 
authorities. Otherwise, you could be getting into far more 
trouble than you plan for your target. 
Throughout this book, we'll be working on the assumptions 
that you're an ethical person, and that your target deserves 
whatever he gets because he’s a bad person. The other side of 
the coin is to avoid harming innocent people while dealing with 
your target. Remember that many gaslighting tactics have side- 
effects, and involve other people in minor roles. 
There's a fine line between involving an innocent person 
tangentially and involving him to the point of harm. The postal 
carrier who delivers a bill to your target isn't likely to suffer. On 
the other hand, instigating a fist-fight between your target and a 
totally innocent person can result in physical harm or legal 
problems for someone who never did anything to offend you. 
By contrast, provoking a non-violent conflict between your tar- 
get and his employer will hurt only your target, because his 
employer can take care of himself. Keep your ethics straight: 
that way, you'll continue to be a better person than the black- 
guard you've selected as a target.    Chapter Two 
Causing Disorientation 
19 

\chapter{Causing Disoreintation}
Chapter Two 
CAUSING 
DISORIENTATION 
The purpose of causing disorientation is to produce a pro- 
found feeling of self-doubt and loss of self-confidence, Spatial 
and kinesthetic disorientation are subtle, but the tactics you can 
use to convince your target that he's losing his memory, and 
perhaps his mind, are blatant and forceful. You can reinforce 
this by inducing the feeling that nothing works for him any- 
more. 
These are subtle tactics, designed both to induce loss of 
self-confidence and to avoid letting him know that he has a 
dedicated enemy, During this first stage, you have to be very 
careful not to leave a signature by using anything resembling 
harassment tactics prematurely, 
Gimmicking Clothing 
and Personal Articles 
If your target wears a hat, and you can find one just like it 
but a quarter-size larger or smaller in a thrift shop, you can 
make him wonder if his head is swelling, or if his skull is 
shrinking. It becomes very easy if your target's headgear is a 
baseball cap with an adjustable plastic backstrap, and he leavesGaslightiag 
20 
it on a shelf, hat rack, or in a locker. Simply change the strap by 
a hole or two when he's not looking. 
Unless your target wears very expensive headgear, it's worth 
the extra cost of buying a new hat, and aging it by wearing it, 
crumpling it, and rolling it in dirt, then cleaning it to match 
your target's hat. Once you've aged it to match your target's hat, 
you can begin substituting it to promote disorientation. 
If your target wears inexpensive, off-the-rack clothing, you 
can confuse him by buying another jacket or two a size too 
small or large. Switch jackets when he's not looking, While 
you're switching the jackets, take an extra minute to look 
through the pockets of his jacket, and place anything you find 
in the appropriate pockets in the new jacket. This will negate 
any suspicion that someone with a similar jacket took his by 
mistake. If you find a key ring or wallet, this is a bonus, and 
we'll discuss how to take advantage of this find later. 
If the target uses a cane, replacing the rubber tips to make it 
appear that he's getting taller or shorter is another way to disori- 
ent him. A telescoping aluminum cane or crutch is even easier 
to adjust for length, usually requiring only a screwdriver or 
wrench to loosen the bolt or nut. Use this tactic carefully, 
changing his cane length in small increments, so as not to make 
it too obvious. 
Move furniture and other objects to disorient your target. 
The more subtle you can be, the better this technique will work. 
If you have access to your target's office or home, the worst 
thing you could do would be to rearrange all of the furniture in 
one move. Your target would immediately realize that someone 
had done it without his permission, and he wouldn't develop the 
creeping self-doubt that results from more subtle methods. He 
might even conclude that someone had played a malicious 
practical joke on him. 
Chapter Two 
Causing Disorientation 
21 
Instead, change one item at a time. Move a chair or lamp, or 
switch his paper baskets to the other side of his desk. If he has 
an adjustable chair, raise or lower it an inch. Next morning, 
your target may not immediately realize that something's out of 
place. Instead, a sense of unease and mild disorientation will 
creep over him as he tries to readjust to the changed surround- 
ings. When he finally notices that something's different, it will 
appear inexplicable. After all, who would sneak into his office 
or home to move a lamp a few inches? 
This works in other settings, as well. If your target's a ma- 
chinist, purloin his micrometer, caliper, or calculator, and wait 
until he's given up searching for it. Then put it back in a 
slightly different place, During the interval, your target may 
have even accused another employee of stealing it, provoking a 
confrontation that will alienate him from fellow workers. When 
the tool reappears, he'll be doubly confused. 
Swapping Calendars 
If you work with your target and he uses his desk calendar 
to record appointments, an excellent way to disorient him is to 
obtain another calendar just like the one on his desk and switch 
them every couple of days. Your target will write in appoint- 
ments, but when you switch calendars he'll write them in the 
second one. When you switch them back, the entries written in 
the second one will be missing, while he'll see others he'd en- 
tered previously, As the entries are all in his handwriting, he 
won't suspect that one of the calendars is spurious, but instead 
will begin to doubt his memory. 
This can be devastating. If your target has a clear recollec- 
tion of having written an appointment a day or two previously, 
and now finds the space blank, he's bound to suffer severe self-  Gaslighting 
22 
doubt as he finds his memory contradicted by the tangible evi- 
dence of a blank space. 
Borrowing His Keys 
Gaining access to your target's keys is important, because if 
you can "borrow" them for long enough to duplicate them, the 
consequences can be enormous. If your target leaves his key- 
ring on his desk, or on his chair, don't be shy about taking them 
as long as nobody sees you do it. The trick is to put them 
someplace else when you bring them back unseen, to gain the 
bonus of making your target think he can't remember where he 
left his key-ring. 
Canceling Telles Machine Cards 
If you have access to his keys, you may also have access to 
his wallet. If so, bring a powerful pocket magnet with you, and 
run it lightly over the magnetic stripe on the back of one of his 
ATM cards, carefully leaving the others alone. Next time he 
tries to use that card in an ATM, the machine will either reject 
or retain it. Either way, it won't work, and he'll have to obtain a 
new card, all the time wondering why his card died on him. 
Vanishing Newspapers 
If you live next door to your target, and he has the news- 
paper delivered, you can add to his feeling that nothing works 
for him anymore. If you can swipe his paper without risk, do it 
a couple of times each week. By itself, this is insignificant, but 
added to other things going wrong in his life, it will add another 
mental hot foot to psych him out. 
Chapter Two 
Causing Disorientation 
23 
The Bottomless Gas Tank 
If you can get the keys to your target's vehicle, make a du- 
plicate of each one, especially if there's a gas cap key, If there's 
an inside release for the gas tank filler cover, you only need the 
key that provides access to the vehicle. 
Never siphon fuel from his tank, as your target would 
merely conclude that someone was stealing his gas. Instead, 
each night add a couple of gallons to his tank. Your target will 
begin to wonder when he's due for another fill-ap. It will never 
occur to him that anyone would add gasoline to his tank, and 
pretty soon he'll begin to doubt his memory. 
Caution: resist the temptation to add molasses, acid, or 
other adulterant to his gas tank, This is crude sabotage, and will 
only warn him that someone has it in for him, big-time. 
The Spurious Sticker 
Service stations usually put a sticker on the door frame with 
each oil change, giving the mileage or date when the next one's 
due. Remove this, substituting a similar one dated a couple of 
months or a couple of thousand miles later. Not only will this 
add to his disorientation, but delaying servicing will increase 
the wear on his vehicle without overt sabotage. 
The Strange Vehicle 
With access to his vehicle, you can carry over the same 
techniques you used in his office. These are especially effective 
if he's the only driver, or if he's the only one to have driven his 
vehicle that day. For example, if he drives to work, get into his 
car and move the seat an inch or two. If he has an adjustable 
steering wheel, readjust it slightly for him. If he normally leaves 
the window down a crack to let hot air out during the summer,Gaslighting 
24 
crank his window up to the top. In winter, lower the window a 
crack. 
Another piece of subtle sabotage has a long-delayed action, 
Open his trunk and let the air out of his spare. This is another 
example of untraceable sabotage that can have serious effects. 
When he needs his spare tire, he'll find it useless, and will end 
up taking it in for a check. Meanwhile, the lack of a working 
spare will cause him serious inconvenience, and he'll be kicking 
himself for not checking his spare tire regularly. 
Don't do anything crude or obvious, such as leaving 
cigarette butts in his car ashtray if he doesn’t smoke. You're 
unlikely to convince him that he has a second personality that 
occasionally emerges to have a smoke. Keep it subtle, and let 
him worry about why his foot has to reach farther to press the 
brake pedal, or why he forgot to tum down the window on the 
hottest day of the year. 
The Dead Battery 
Another way to convince your target that he's losing his 
mental capacity is to tum on his headlights while his vehicle's 
parked. This works only if your target normally drives with 
headlights on, commutes to work before daylight, or drives 
somewhere after dark. You simply unlock his door and turn on 
his lights, to ran down his battery. 
At work, you appear to be helpful, parking your car near his 
so that you notice his distress at quitting time, and to ingratiate 
yourself with him, you offer him a jump-start with your cables. 
After the second or third occurrence, you can suggest to him 
that his memory is slipping. 
Chapter Two 
Causing Disorientation 
25 
The Moving Car 
Another disorientation tactic you can use if you have the 
keys to his vehicle is to move it a couple of times a week. When 
you see him park it at work, wait until he's inside the building 
and move it to another parking spot. When he emerges, he'll be 
dismayed to find it gone, the moment of panic replaced by 
puzzlement when he sees it in a place he doesn't remember 
leaving it. 
To make sure he doesn't begin to suspect this is being done 
by a fellow employee, follow him to a supermarket or shopping 
center, Once he's inside, relocate his car a couple of rows away. 
If his home doesn't have a garage, and he parks his car in his 
driveway or on the street, move his car one night. 
Begin this program slowly, relocating his vehicle once a 
week, and step up your campaign later. For best results, move it 
only one or two slots away for the first week, then increase the 
distance until he becomes convinced that his memory is as full 
of holes as the hull of the Titanic. A bonus is that your tactic 
can have repercussions, taking advantage of the multiplier ef- 
fect. Two possible scenarios are: 
1. Your target doesn't find his car where he left it when he 
comes out of a shopping mall. He reports it as stolen to po- 
lice, and when a police officer arrives to take a report, he 
finds it two rows away. This is likely, because police offi- 
cers have had many "stolen car" reports from people who 
were merely forgetful, and often take a quick tour of the 
parking lot to see if the vehicle is on the premises. 
2. Ifa family member, most likely his wife, also drives the ve- 
hicle, relocating it can provoke a confrontation between 
them. He'll accuse her of not telling him she had parked the 
car elsewhere, which she'll deny. Despite the repeated de-  Gaslighting 
26 
nials, he'll continue to find his car elsewhere, which won't 
improve his disposition one bit. 
The Fake Key 
Jamming your target's home or vehicle lock with a tooth- 
pick or super glue is nasty and effective, but it would be 
counter-productive for your purposes. Your target would im- 
mediately know that someone had played a nasty trick on him, 
and would be angry at whomever he suspected. Instead, do 
something that will leave him confused and disoriented. 
If you have access to your target's key ring, obtain a house 
key that closely resembles his, then swap them. He'll work up a 
sweat wondering why a key that looks so familiar doesn't work 
in his lock. He will have to call a locksmith to straighten out his 
problem, and the expense will far exceed whatever it cost you to 
have the fake key made. You can do the same for his vehicle. 
One quick and dirty way to do this if you can't "borrow" 
your target's key for more than a couple of minutes is to use a 
needle file to file a few thousandths of an inch off one of the 
key's teeth. This will make the key unable to open any lock for 
which it's been cut. 
An even quicker way is to break off the tip of a pencil in his 
cylinder lock. This will jam the lock temporarily, but the pencil 
lip will crumble into powdered graphite as he continues his ef- 
forts to insert the key. As graphite is a normal lock lubricant, 
and not a foreign substance, even a locksmith won't find any- 
thing to indicate tampering. 
If you have elementary locksmithing skills, you can take 
this a step further. Instead of substituting a key, remove the 
cylinder of his front door lock and replace one tumbler with a 
slightly longer or shorter one, so that his key will no longer 
work. If his front door key also works for the back door or the 
Chapter Two 
Causing Disorientation 
27 
storage room, leave those locks alone. Your target will now 
have a key that works perfectly on every lock but the one on his 
front door. When he calls in a locksmith, and finds out that one 
tumbler in the door lock doesn't fit his key, this explanation will 
be incomprehensible and confusing. 
Pledging To: Charities 
Various causes regularly hold telethons and other cam- 
paigns to solicit contributions. One often used technique is to 
encourage viewers to telephone their pledges. Public television 
stations solicit pledges one or more times per year, To make 
this work, you simply telephone a pledge in your client's name 
to every telethon you can, The campaign operators will send 
your target a letter thanking him for his pledge, and an envelope 
in which to mail his check. Your target, of course, won't re- 
member pledging anything. If he has, by coincidence, pledged a 
sum to that charity, it will probably be a different amount, and 
he'll conclude that someone else made a mistake. When the 
collection request with the correct amount arrives, it will appear 
confusing. 
Party Time 
If you're lucky enough to be close to your target, you may 
be able to disrupt his plans for a social gathering. A few exam- 
ples will illustrate some tactics you can use: 
If your target orders pizzas to be delivered at a certain time, 
and you know the pizza shop involved, you can disrupt his 
plans by a follow-up call. Don't do anything crude, such as can- 
celing the order. Instead, change it, Telephone the pizza shop 
and order a change in the number or types of pizzas, or the time 
of delivery. If the pizzas arrive an hour early, they will be cold 
by the time guests arrive. If late, your target will wonder where  [ Gaslighting 
28 
his pizzas are, and may have to try to explain their absence to 
his guests, 
If the invitations were verbal, and you're one of the guests, 
you arrive an hour early, and insist that that is the time he told 
you to come. At this point, an accomplice can be very helpful, 
if he also arrives early and tells the target the same thing. 
With an accomplice, you can be even bolder. Telephone 
your target a few days before the gathering and tell him that you 
can't make it, then show up anyway, Your accomplice arrives 
with you, and if your target mentions that he hadn't been ex- 
pecting you because you said you weren't coming, your ac- 
complice states that this can't be so because you had planned to 
share the ride and attend together. This version is even more 
credible and convincing if the story is that one of you is the 
“designated driver" who will avoid drinking alcohol, following 
today's practice. 
A Matter of Taste 
As a guest in your target's home, you'll have priceless access 
which you can exploit cleverly, Apart from copying his keys 
and other preparations, you can gimmick his condiments 
lightly, to make him think that his taste buds are deceiving him. 
One way to do this is a variant on the old trick of replacing the 
sugar in his sugar bowl. Substituting salt would be entirely tao 
obvious, but adding a small amount, just enough to make his 
coffee taste slightly strange, is enough. Adding a teaspoon of 
sugar to his salt-shaker will also produce a vague off-taste, a 
suggestion that his senses are fooling him. This provides one 
more increment of sensory disorientation. 
A light sprinkling of finely powdered salt in a container of 
ice cream will also produce an unusual taste. For best effect, 
make sure the container you adulterate is partly empty. If you 
Chapter Two 
Causing Disorientation 
29 
tamper with a fresh one, he might just conclude that the manu- 
facturer had produced a bad batch. 
If your target is a dedicated coffee lover, you can tamper 
with his coffee blend to make it taste unfamiliar. Most super- 
markets sell "gourmet" coffee blends, many with artificial 
flavorings such as mocha, vanilla cream, mint, or hazelnut, Add 
a teaspoon of one of these flavored variants to his canister of 
coffee and shake well to blend it thoroughly. Again, be discreet 
and add only enough to change the taste subtly, to avoid mak- 
ing it obvious. 
If your target uses cologne or after-shave lotion, this pro- 
vides another opportunity for increasing his disorientation. 
Obtain a bottle of another brand, similar in color but with a 
very different odor. Add a few drops to his regular bottle, using 
just enough to change the odor without making the change 
glaringly obvious. 
TIP! Many cologne and after-shave bottles have narrow 
shaker openings, making pouring practically impossible. Obtain 
a plastic syringe with a nozzle that will slip inside the aperture 
of your target's bottle, You should not need the needle. Even in 
states that restrict the sale of hypodermic syringes, you can ob- 
tain plastic syringe bodies in pet stores, where they're sold as 
implements for administering medicine to pets. 
Simple Addition 
You can produce severe disorientation if your target is your 
roommate or if you have legitimate access to his home. If he 
keeps a half-gallon of ice cream in his freezer, substitute an al- 
most full container of the same brand and flavor when his is 
almost empty. Likewise, add coffee to an almost empty coffee 
can. The only limitation you must observe is to restrict your 
tampering to things he uses exclusively, Do not, for example,  Gaslighting 
put a fresh toilet paper roll in place of one that’s almost fin- 
Paying Back A Loan 
This is one of the few tactics in this book that's absolutely 
fool-proof, because it will work with stunning effect every time. 
You approach your target and hand him a ten-dollar bill, 
thanking him for the loan of the money. When he appears con- 
fused, you gently "remind" him that he had, indeed, lent you the 
ten dollars the previous week. Your target won't be able to un- 
derstand how he could have forgotten that he'd lent you the 
money, and it will be inconceivable that anyone would pay back money he didn't owe. The effect of this tactic increases if 
you have an accomplice who does the same thing to him the 
following month, 
Christmas Cards 
One way to cause your target to puzzle over the gaps in his memory 1s to send him Christmas cards from people he doesn't know. Have several accomplices sign Christmas cards with first names only, and mail them to your target without a return ad- dress. As return addresses on Christmas cards are much less common than on business mail, he's unlikely to think the send- ers wrote the cards by mistake, and will wonder which of his friends he's forgotten. Using very common first names, such as "John" and "Bill" will add to the confusion. 
Adding a personal message to the bottom of one or two cards will enhance the confusion. A line reading, “How did you 
like the cuff-links?," or "Hope the shirt was your size," will have your target trying to connect the card with presents he’s re- ceived, or wondering what happened to the gift mentioned in the card. 
Chapter Two 
Causing Disorientation 
31 
Old Army Buddy 
If you know that your target was in the armed services, one 
way to make him ponder which of his old buddies he’s forgot- 
ten is to telephone his home or office while he's away, give a 
name, and leave a message that you're an old army or navy 
buddy and that you intend to take him up on his offer to look 
him up if you ever got to town. You don't leave an address or 
phone number, explaining that you're just passing through at 
the moment, but will be returning in the other direction next 
week, and will try to contact him again. 
The Phantom Acquaintance 
Another way to convince your target that his memory is 
slipping requires an accomplice who has never met your target. 
To carry this out successfully, you must have solid information 
about your target, you must be a "friend" or fellow employee, 
and you must brief your accomplice thoroughly, 
Your accomplice, Charlie, approaches your target at a busi- 
ness or social gathering and greets him warmly, referring to a 
meeting they had the previous week or month. Your target may 
at first assume that this is a case of mistaken identity, but your 
accomplice's next words suggest that they had actually met and 
had a conversation: 
Charlie: Hi, Harry, good to see you again, Did you catch any 
fish last weekend? 
Target: Huh? I'm not sure we've met. 
Charlie: Sure we did, after the sales meeting last week. You 
told me you were going up to the lake for the weekend. 
At this point, you approach to tip the balance:    Gaslighting 
32 
You: Hi, Harry, Charlie. I see you've met again, Harry, Char- 
lie was reaily interested in your fishing, ‘cause he fishes 
himself whenever he can get away, 
Reinforcing Charlie's acquaintanceship is crucial, as it 
makes it impossible for your target to ignore. With you insisting 
that you introduced him to Charlie, and Charlie's referring to 
his fishing trip, there's no way Harry can ignore this, or pass it 
off as a case of mistaken identity. 
Canceled Appointments 
Another way to give your target a mental hot foot and pro- 
mote disorientation is to telephone him when he's out of the of- 
fice or away from home, leaving a message with a fellow em- 
ployee or with his wife that you won't be able to meet with him 
as agreed. Of course, you leave an unfamiliar name, to confuse 
him further. An additional twist, if you want to make him waste 
some of his time, is to leave a telephone number. 
The number you leave is not just a number you picked at 
random, either. Copy the number of a pay telephone inside a 
Store you know closes at six, and leave word that your target 
can reach you there any time after seven. You do not choose the 
number of a person or business, because there might be an an- 
swering machine. With a pay phone, unless someone remains 
late, there will never be any answer when he tries the number, If 
he tries to dial the number outside the hours you specified, the 
tactic will still be effective because he'll be forced to conclude 
that his wife or fellow employee copied the number incorrectly. 
This can provoke a confrontation, especially if your target be- 
lieves that the call was important. 
Chapter Two 
Causing Disorientation 
33 
Occupational Insecurity 
A major point laid out in the chapter on basic tactics is that 
the more you know about your target, the more effectively you 
can strike at him. No tactic brings out this point better than the 
fake employment ad. 
If you know where your target works, exactly what he does, 
and if you also know that your target regularly scans the “help 
wanted” ads, this knowledge offers you a priceless opportunity 
to strike a blow at his peace of mind. The tactic is to place a 
classified ad for your target's job in the newspaper he reads. Be 
as specific as you can, listing the company and the exact posi- 
tion. 
This works best if your target's position is unique in the 
company. He's not going to worry much if he sees an ad for as- 
sembly-line workers and he's one of 200 in that position. How- 
ever, if he's the Purchasing Director, or Comptroller, he's on the 
hot seat. Note that this tactic will be terrifyingly effective if 
your target already feels insecure about his position, or has re- 
cently had a confrontation with the boss. 
This is a perfect example of the need to remain alert to sud- 
den opportunities. You may not have even considered a fake 
employment ad in your original plan, but if, for reasons entirely 
unconnected with you, your target has a heated, door-slamming 
confrontation with his boss, the opportunity this provides is too 
good to ignore. 
The beauty of this plan is that, even if he confronts the 
situation squarely and asks his boss if he's going to be replaced, 
the most vehement denial will not give him peace of mind, 
Employers customarily don't provide an outgoing employee 
with advance warning. In some companies, fellow employees 
know of the impending execution through the rumor mill, but 
the victim finds out only when his supervisor takes him for the  Gaalighting 
34 
short walk down the hall and furtively presses the final pay- 
check into his sweaty hand. 
The only explanation that would put his mind at ease is that 
he's up for promotion, so of course the company has to find 
someone to fill his slot. Fat chance. 
Mail Games 
One subtle way to make your target appear strange to corre- 
spondents is to add an extra stamp to several of his letters as 
often as you can. Extra postage works best if the envelope con- 
tains only one sheet of paper, making it obvious that the extra 
stamps are wasted. This will make it appear that he used very 
poor judgment in sticking on excessive postage. In fact, recipi- 
ents of these over-franked letters may never mention it to your 
target, although they'll privately think he's beginning to get 
flaky. Even if one recipient does bring up the subject, this can 
never point a finger directly at you. 
Vanishing Mail 
It's not a crime to put too many stamps on an envelope, but 
tampering with the mail is very illegal, so be extremely careful 
if you decide to make off with even one piece of your target's 
incoming or outgoing mail, Intercepting his mail can be very 
effective because it’s hard to pin down responsibility for a letter 
that never gets to its destination. You have to use this tactic in 
moderation, though, to avoid the appearance of a deliberate ef- 
fort by a persecutor. Let's take a look at a couple of scenarios 
that illustrate how vanishing mail can be devastating: 
You know that your target is buying a new house. 
You see an envelope from the real estate agent in his 
incoming mail, and purloin it, carefully leaving all the 
other mail. The application, or contract, never arrives, 
Chapter Two 
Causing Disorientation 
and your target will be contacting the agent, who will 
insist that he sent it. In some cases, the results will be 
inconsequential, as the agent can send another con- 
tract, but if time is of the essence, your target will lose 
valuable days before he realizes that something's gone 
seriously wrong. 
Another: 
Your target is in the habit of paying his bills by mail, 
and leaving them in the office outgoing mail bin. Rif- 
fling through the envelopes, you see one addressed to 
the local utility, the mortgage company, or the credit 
card company, and you take it. The result will be that 
your target will receive a notice of non-payment, and 
his credit rating can suffer. Your client may insist that 
he sent the payment, but the creditor has heard all 
possible excuses from delinquents and deadbeats, and 
won't buy his story. 
This can be especially devastating if the envelope 
contains an auto insurance payment, and his policy 
lapses because of non-payment. Equally troublesome is 
a license plate renewal. Your target may be driving on 
expired plates before he catches up with the situation, 
If the cops stop him, they're unlikely to take his word 
that he sent in his renewal. 
Yet another: 
Your target signs a purchase order, contract, or 
other legal document, and you're able to obtain a copy 
of the original form. You pick his envelope from the 
outgoing mail, open it to retrieve that paperwork, type 
the relevant information on your spare copy of the 
document, and seal it in another envelope, without his 35  Gaslighting 
36 
signature. You mail it to the intended recipient, who 
will then be forced to return the paperwork to your tar- 
get with a request that he sign it, giving him yet another 
mental hotfoot. 
Don't try to be funny and sign "Adolph Hitler" or "Hillary 
Clinton" to such documents. This will only tip your hand, be- 
cause both your target and the other party will know that a 
prankster or more sinister influence is at work. 
Switching Envelope Contents 
Yet another way to produce aggravation and self-doubt for 
your target arises if he mails several payments on the same day. 
Using 4 wet sponge, you open two envelopes containing 
checks, and switch them. Thus, the Internal Revenue Service 
receives a check made out to the electric company, and the 
electric company receives the IRS check. The electric company 
will not accept the check made out to the IRS, which in any 
case is likely to be excessive, but the IRS sometimes over- 
stamps the payee with its own stamp, depositing the check as 
partial payment and billing your target for the balance. This will 
cause him extra problems with his checkbook balance. Again, 
this is very unlikely to be traced to you because putting the 
wrong check in an envelope isn't a terribly uncommon mistake. 
The Missing Check 
Another way to cause your target a delayed-action mental 
hotfoot presents itself if you have access to his checkbook. Re- 
move one check from near the bottom of the pad, tear it up, and 
flush it down the toilet. When your target gets to the one below, 
he may notice the skip in number sequence, and begin racking 
his brain trying to remember a check he’s sure he must have 
written but forgot to record. If he doesn't notice the gap, he may 
Chapter Two 
Causing Disorientation 
37 
get a reminder if his bank lists gaps in check sequence on its 
statements. 
Laundry Time 
If you have access to your target's laundry hamper, note the 
contents and select a bright color from the items inside the 
hamper. Buy a package of dye the same color, and at the next 
opportunity, pour the contents of the dye envelope into a pocket 
of the apparel you've chosen, When this goes in the wash, it 
will appear as if the color has “run” to stain all of the contents 
of the washer. 
The Bottom Line 
A well-planned and forcefully executed program of psycho- 
logical tactics can shatter your target's self-confidence, After a 
series of disorienting incidents, he'll feel as secure as the cap- 
tain of the Titanic after the impact.    Chapter Three 
Building Paranoia 
39 

\chapter{Building Paranoia}
Chapter Three 
BUILDING 
PARANOIA 
Paranoia, for our purposes, means a feeling of persecution, 
and is an important step in your program of psychological 
warfare. Promoting such a feeling can make the target's anxiety 
level shoot up to the ceiling, and when your target's tense and 
anxious, his judgment will suffer. The purpose is to build a 
diffuse anxiety which cannot focus on any individual, and espe- 
cially not on you. 
Note that if you're not too subtle about it, this can let your 
target know that one particular individual is out to get him, 
which is counter-productive. This is why you should reserve 
some of these tactics until you've already got him confused. 
The main fact you've got working for you in building para- 
noia is that most people are already paranoid or fearful for very 
realistic reasons. Your target isn't the only hateful person on 
Earth. There exist many unfriendly neighbors and fellow em- 
ployees, and some are predatory by nature. It's very realistic to 
be apprehensive about such people, just as it is to worry about 
the "crazies" in our society who open fire in a post office or 
restaurant. Working to enhance your target's paranoia is easy 
because the nature of our society is on your side. In any case, 
the dictum, "Paranoids have enemies too," works very well 
when you're gaslighting your target.Gaslighting 
40 
The Anonymous Note 
One perfect way to build anxiety is to go out to the com- 
pany parking lot and leave an anonymous note on your target's 
windshield. The note reads: 
Attention: This is my spat. If you park here again, I'll 
Slash your tires! 
Unsigned, it provides no information regarding the identity 
of the writer. Your target may bring the note to the attention of 
the company's security department, if there is one, but will 
spend several days worrying whether his tires will be slashed in 
reprisal. The reason? Let's look at this closely: 
The note indicates that someone thinks that the target 
parked in "his" spot, something obviously incorrect. This sug- 
gests that the writer isn't totally rational, and is extremely angry. 
The target may wonder if the writer is planning to do something 
to his car anyway, in reprisal. 
A particularly effective variant of this tactic becomes pos- 
sible if you have the keys to your target's car, because you don't 
have to create a phantom enemy, After he parks it, you move 
his vehicle to a spot assigned to someone else. There's no guar- 
antee that the person offended will write a nasty note, but if you 
park the car in a spot reserved for the company president, the 
repercussions can be worse. Your target may recéive a polite 
reprimand the first time, but after you relocate his car the third 
or fourth time, there will be fireworks. You can be sure that the 
boss will impose severities for this persistent and arrogant dis- 
regard of his prerogative. Your target's fervent denials will be 
useless, and may even aggravate the situation. 
Chapter Three 
Building Paranoia 
41 
Whispering Sneers 
Another way to foster your target's paranoia is to make it 
clear that people are, indeed, talking about him. For this tactic, 
an accomplice or two are essential. When your target is in the 
room, whisper to your accomplice, both of you looking in the 
target's direction to suggest that he's the topic of the conversa- 
tion. Sneering while you whisper will suggest that he's being 
mocked, causing a very uncomfortable feeling. 
It's necessary to use finesse to handle a possible confronta- 
tion, such as your target's turning on the people sneering and 
asking them if they're talking about him. A bold counter-attack 
will smash his self-esteem: 
"What's the matter with you? Think you're important 
enough for people to talk about you? Getting paranoid 
or something?" 
If you can get several accomplices to do this, the effect will 
be to diffuse his attention. Instead of focusing on one or two 
people who don't like him, he'll get the feeling that mockery 
and ridicule of him are widespread. 
The Anonymous Accusation 
Hitting your target with an anonymous accusation works 
best if he has a guilty secret, and you know what it is. An ex- 
ample is a married man’s office romance. Leaving an accusing 
message on his voice-mail will make his blood pressure rise: 
"Does your wife know what you're doing with Mary 
Jones when you take a three-hour lunch with her?” 
Another scenario is the executive who regularly raids the 
petty cash:Gaslighting 
42 
"Does the boss know about those fake vouchers you've 
been filling out? What if he knew that you took the bus 
when you put in a chit for cab fare?" 
These can create unlimited anxiety, because your target has 
no way of knowing how many people know his guilty secret. 
He does, however, know that people love to gossip, and if he 
has offended many people, can anticipate that they'll spread the 
news. 
Let's look at some possibilities you can create even if you 
don't have any dirt on him. Some real-life situations generate 
suspicion, even if the person is totally innocent. Let's look at 
how you can create something out of nothing if your target is a 
doctor: 
"Mrs. Jones was alone with you in your office for an 
hour last Friday. Does her husband know what's going 
on? What about your wife?" 
This is particularly worthwhile because of recent scandals 
involving doctors who had sex with their patients. Another 
prospect is the clergyman or boy scout leader, 
The scouting movement is particularly sensitive to this sort 
of accusation because adult leaders who molest members of 
their troops are not uncommon. This is why many scout troops 
guard against this by requiring that no adult be alone with a 
child, and adult leaders work in pairs or groups. Still, a scout 
leader may occasionally be unavoidably alone with a child for 
awhile, and if you become aware of such a case, you can fabri- 
cale a Situation that suggests guilt: 
"People have been wondering why you took Tommy 
Jones off by yourself last Friday. Do his parents know 
what you did with him?” 
Chapter Three 
Building Paranoia 
43 
If your target doesn't have either voice mail or an answering 
machine, you can obtain the same effect with an anonymous 
note to his home or workplace. One tip: don't seal the envelope. 
Make it easy for the office snoop or a nosy relative. 
Mind-Bending 
It's remarkably easy in today's tense world to make someone 
think that people are talking about him, or have it in for him. 
Some people are predisposed to this outlook, and your tactics 
only intensify their suspicion and distrust.as 
=4 
\chapter{Destroying Your Target's Reputation}
Chapter Four 
Destroying Your Target's Reputation 
45 
Chapter Four 
DESTROYING YOUR 
TARGET'S REPUTATION 
Extending the concept of gaslighting involves not only 
making the target doubt himself and his sanity, but causing his 
relatives, friends, and associates to wonder about him as well. 
The legal doctrine of an employer searching an employee's 
desk or locker for evidence of illegal activity is well established 
in this country. There are searches for stolen company property 
and illegal drugs. Certain companies have their security guards 
examine any packages or briefcases employees carry out the 
door. This provides the key to sabotaging your target's reputa- 
tion, and lowering his esteem in the eyes of fellow employees. 
Offensive Publications 
If normal practice calls for a security guard to search every 
briefcase or package, smuggle a homosexually-oriented or 
bondage magazine into his briefcase. The guard's eyebrows will 
surely go up, and the find may become the subject of gossip. 
Placing a perverted magazine in your target's desk can also 
be damaging. One way to arrange for it to be "found" is to have 
another employee find it. One way to contrive this innocently 
comes if a fellow employee asks you for a copy of a report or 
other paperwork. You can tell him that you lent it to Harry Tar-Gaslighting 
46 
get, and that it's probably in his desk. If your target is absent 
that day, he won't be around to deny that he has it, and the other 
person may look for it in his desk. Voila! He finds the magazine 
and the rumor mill begins. 
Let's note that the publication need not be homosexually 
oriented. In fact, this might be useless in certain workplaces. In 
such cases a magazine covering bondage, whipping, or some 
other form of sadism may do the trick. In some places, inter- 
racial sex is politically incorrect. The key is that the publication 
must be contrary to the ethic or culture of the employer or 
fellow employees. 
It doesn't have to involve sex. Someone employed by a lib- 
eral politician can suffer if you plant a copy of a right-wing 
publication in his desk. An abortion clinic employee will suffer 
embarrassment if a copy of a publication by the religious right 
surfaces in his possession. 
If your target works for the American Civil Liberties Union 
or a similar organization, write to the Ku Klux Klan or the 
White Citizens’ Council for information and a membership 
application form, using his name and work address. These or- 
ganizations use a boldly-printed return address on the envelope 
as a form of advertising, and anyone at your target's office who 
handles the mail will probably notice it. 
Pare ae 
Rumors, whether based on fact or pure fiction, can be very 
damaging, but they can also kick back at the person starting 
them, Instigating a rumor without ending up on the business 
end of a lawsuit for defamation requires discretion and a very 
delicate touch, Don't even try this unless you're sure of success, 
The essence of defaming your target with a rumor is that it 
not get back to him. In many locales and workplaces, the domi- 
Chapter Four 
Destroying Your Target's Reputation 
47 
nant ethic is avoiding confrontations, and accusations will cir- 
culate behind a person's back without ever reaching him. One 
way to start a damaging rumor is if you have a female accom- 
plice in the workplace or in your target's social circle. A second 
requirement, not absolutely essential, is that your target be un- 
married. 
Your female accomplice tells others in the group that your 
target has asked her for a date, She then relates one of several 
versions of what happened. You can really use your imagina- 
tion in thinking up stories: 
e Your target asked her to go to a funeral for their very first 
date, saying he enjoys attending funerals. 
e He turned out to have kinky sexual tastes, such as whips 
and chains, or threesomes. 
e He invited her to his home, where contrary to her expecta- 
tions, they spent the entire time watching kinky sex video- 
tapes. 
e Your target tried to have sex with her, but was impotent. 
He was inexcusably rude and inconsiderate. After sex, she 
asked him when they would meet again, and his reply was: 
"Not necessary, my dear, I've already had you." 
e He turned out to be secretly married, 
The effectiveness of these rumors will be enhanced if your 
target is already thought of as a geek or jerk by his associates. 
They won't mention the rumors to him, but simply add the 
damaging allegations to their store of knowledge about him, 
while enjoying a few Laughs behind his back. 
There is another type of rumor, "disinformation," intended 
to get back to the target. Disinformation is the ultimate mind- 
game, designed to mislead the target, and we'll delve deeply 
into the tactics and uses of disinformation in a later chapter.  Gaslighting 
48 
Noles on Toilet Walls 
Writing a note on a toilet wall, previously considered a 
practical joke, now has more serious consequences if you add 
your target's name and telephone number to them. One man 
was arrested after someone noticed a note written on the toilet 
wall in a K-Mart. The note invited boys to call a telephone 
number for sexual contacts, and a police officer with a young- 
sounding voice called the listed number. Police recorded the 
conversation, during which the writer incriminated himself, and 
arrested him on one count of "attempted" sexual contact with a 
minor.! 
Whiting notes on toilet walls won't get your target con- 
victed, but it may draw the attention of the police to him. He 
may deny any perverted proclivities to a police undercover 
agent who phones, but that may not stop the investigation. Po- 
lice know that pedophiles often seek out situations in which 
they come into contact with minors, and if your target is a 
teacher or scout troop leader, he'll receive close scrutiny from 
the sex crimes detail because he fits the profile of a sex of- 
fender. 
Any close police investigation causes ripples. Police offi- 
cers who interview friends and neighbors may request that they 
keep the interview confidential, but some people can't resist the 
temptation to gossip about anything involving scandal. More 
importantly, your target's employer, if interviewed by police, 
may hold it against him. It doesn't matter at all if the police in- 
vestigation "clears" your target. Both police and employer will 
probably assume that, although no hard evidence surfaced this 
time, the target's probably got something to hide. 
If he's a teacher without tenure, his contract may not be re- 
newed next year. If in another occupation, he may be on the 
Chapter Four 
Destroying Your Target's Reputation 
49 
execution list for the next down-sizing. Whatever else may 
happen, promotion will henceforth be out of the question. 
Naughty Magazines 
If your target has an office with a waiting room, the odds 
are overwhelming that he'll have magazines on a table for his 
clients’ perusal. "Salting" a few naughty publications among the 
legitimate ones will raise eyebrows. 
An architect, for example, is likely to have a stack of archi- 
tectural publications laying on the table. A few sexually-ori- 
ented magazines, the nastier the better, under the top ones will 
soon come to the surface. 
This can be devastating if your target's occupation is one 
that demands rigorous morality. In some locales, nobody cares 
if an architect or interior decorator they hire is homosexual, but 
a pediatrician or clergyman would be sorely embarrassed if his 
patients or members of his congregation were to discover publi- 
cations that called his morality into question. 
How do you reinforce the impact, and make it impossible 
for your target to deny that the publications are his? It's very 
simple, You buy white stick-on labels at a stationery outlet and 
type your target's name and address on them before sticking 
them on the publications obtained from an adult or sex shop. 
To cause him problems at home, if he's married, just buy 
him a subscription to a raunchy magazine, the sort found only 
in adult bookstores. Don't bother with Playboy or Penthouse, 
even though these have subscription cards that allow you to pay 
later, unless his wife is very puritanical. The really raunchy 
mags require the subscriber to enclose payment with his order, 
so this will cost you twenty or thirty dollars. The result will be 
worth the money! 
Yet another way to cause your target severe embarrassment 
with his neighbors is to fake a subscription for him. You buy a  Gaslighting 
50 
homosexual or sadism magazine at an adult shop, type a white 
label, and stick the Jabel to a plain brown envelope. Before 
dropping the envelope in the mailbox, you tear it open enough 
so that anyone picking it up can see the type of publication it 
contains. 
The key to this tactic is the label, You type your target's 
name correctly, but use the address of a neighbor several houses 
away. If the mail carrier doesn't know everyone on his route, 
and delivers strictly according to the address, the envelope will 
end up in the neighbor's mailbox. The rest is inevitable. 
This will also work if you address the envelope to your tar- 
get's workplace, as long as you make sure to tear the envelope. 
When you do this, it doesn't matter what sort of work your tar- 
get performs, or if he has a secretary or an office, Several fellow 
employees will see the raunchy magazine, and the news will 
spread. 
Yet another way to produce this result is to send letters re- 
questing their catalogs to various firms advertising in raunchy 
magazines. Provide your target's name and address, and sign his 
name, adding the statement, "| am over 21." Ads in such publi- 
cations are for sleazy cross-dresser clothing, leather goods, dil- 
does, vibrators, cock-rings, tit-clamps, and other raunchy sexual 
parapheralia. Not all catalogs come in plain brown envelopes. 
Condoms and Other Items 
Planting condoms where they're likely to be found by your 
target's fellow employees, girlfriend, or wife can be devastating. 
You have to choose your target judiciously, however. A bache- 
lor would probably not be embarrassed if a fellow employee 
were to see a box of condoms when he opened his desk drawer. 
However, a wife is another matter, especially the wife of a 
Chapter Four 
Destroying Your Target's Reputation 
51 
doctor or clergyman. Planting condoms can be very easy in 
some instances, 
In the southern states, motorists often leave a window open 
a crack during the summer months, to let out the hot air, when 
they park their cars. A condom will easily slip through the 
crack, The method is to slip a condom through the passenger's 
side window, if it’s open, as it won't work as well on the driver's 
side. If the target's wife notices the condom on the seat or floor, 
she'll have some pointed questions to ask him! A used condom 
will have an even better effect! 
Anyone who leaves his car unlocked leaves himself open to 
more raunchy tactics. Leaving a well-worn and smelly bra in 
the back seat is one way of provoking accusatory questions 
from your target's wife. Dirty panties, poop stains and all, are 
even more spectacular. 
If you and your target go on a business trip together, this 
provides a perfect opportunity for framing him. If you know 
that his wife packs and unpacks his suitcase, slip a bar of soap 
wrapped with another hotel's label in his bag on the last day of 
your trip. The wife will wonder why he told her he would be at 
the Holiday Inn in Dayton when she turns up a bar of soap from 
a Best Western in Chicago! 
If you're determined to be raunchy, slip a pair of dirty 
women's panties into his suitcase just before the return home. 
His wife will get a big surprise when she unpacks hubby's case! 
If your target calls on you to substantiate his whereabouts 
with his wife, back him up in the certainty she won't believe 
you. Many women believe that men stick together to cover each 
other's infidelities. You'll have had the satisfaction of sticking a 
knife in his back while appearing to support him all the way!  Gaslighting 
52 
Instrument of Evil 
The telephone offers an entry into your target's life which 
you can exploit with devastating effectiveness. The following 
tactics work better if your target's number is unlisted: 
A male accomplice telephones your target's home at a time 
when you know his wife is out, and asks to speak with her. He 
doesn't give his name, nor a number where he can be reached, 
and acts embarrassed that the husband picked up the phone. 
The second version works in reverse. An anonymous female 
telephones while the husband is out, asking for him, and hangs 
up abruptly after the wife answers. Sorry, wrong number! 
This tactic is most practical if you're a neighbor, and can 
see exactly when your target or his wife leaves their home. Al- 
ternately, if you're a co-worker, you know when your target 
shows up for work. Otherwise, you either have to "stake out” 
their residence, which attracts unwanted attention, or you have 
to take the risk that the person you're calling will actually be 
home. 
A very devastating telephone technique holds absolutely no 
risk for you, and works every time. If you know that your tar- 
get's having an affair, discreetly shadow him, and when he's al- 
most home, dial his number. Hang up the moment his wife an- 
swers. This ensures that his wife will be awake when he enters, 
and aware of how late he arrived. If he'd promised to be home 
early, and depended on her being asleep when he finally 
crawled into bed, this technique will blow his plan sky-high. If 
he's stinking drunk upon his return home, so much the better. 
Swingers’ Clubs 
Employing this tactic illustrates the need to know as much 
as possible about your target, his habits, and personal peculi- 
Chapter Four 
Destroying Your Target's Reputation 
53 
arities. Using a swinger's club, or classified dating service, is 
one way to deliver a blow to your target's reputation. 
The success of this tactic depends on your target's situation. 
If he's married and church-going, any proof of his adultery will 
have severe repercussions. In many environments, any indica- 
tion of perversion will be a black mark against him. In some 
situations, the scandal will be serious enough to cause him to 
leave his job, or even to leave town. 
You may discover, or merely suspect, that your target places 
or answers advertisements in "singles" magazines or publica- 
tions devoted to sexual highways and byways. This may happen 
because he tells you, or you may see a copy of the publication 
in his desk, briefcase, home, or car. 
If you do, don't rush to make others aware of his proclivi- 
ties. Instead, note which, if any, advertisements he's marked, to 
gain an idea of his range of "interests." The next step is to place 
several similarly-worded advertisements in the same publica- 
tion, using a blind address. 
This can be a P.O, box, or a mail drop, but many publica- 
tions today provide blind addresses for their advertisers. Re- 
sponders address their replies to "Box xxx," care of the publi- 
cation. 
The nature of the reply depends on the type of publication, 
your target's proclivities, and the way you worded your ad. It 
may simply be a letter describing himself in very flattering 
terms, or it may be even more explicit. 
Try to obtain a frontal nude photo of your target, which can 
be useful for other tactics. Trolling for photos depends upon 
how you word your ad. Some ads, in the more explicit publica- 
tions, insist on a photograph, which, in this context, usually 
means a nude frontal photograph, Some state clearly; "No 
photo, no answer," or "Replies with photos get answered first."  Gaslighting 
54 
With a little luck, your ads will pull in a reply from your 
target. The next step is to make photocopies of his reply, and 
copies of any photographs he may include. Finally, you mail a 
copy to his wife, employer, his neighbors, and his fellow em- 
ployees to devastate his reputation. 
Lost and Found 
Bibliophiles often have stickers inside the front covers of 
each of their books, promising a reward to the person who finds 
and returns it. Naturally, the sticker lists the owner's name and 
address, and sometimes even his telephone number. You can 
employ this as a gaslighting tactic against your target. First, get 
a pad of stickers printed: 
This book is valuable. There is a ten dollar reward for 
its return. If found, please return to: 
Harry Target 
132] Bleep Avenue 
Sleazeville, USA 
Next, buy some raunchy porno novels with racy titles such 
as: Hot Panis Homo, The Sadistic Japanese Submarine Cap- 
tain, Paddling For Joy, Handcuff Lover, Going Down on Rick. 
The more perverse and nasty they are, the more successful 
your plan will be. For those of you totally unfamiliar with porno 
paperbacks, the era of plain covers is over. Today, most have 
explicit cover illustrations to capture a browser's interest. 
Don't waste your money on new ones, though. Instead, buy 
used books from a second-hand bookstore, for economy. In any 
case, old and worn books will do better because they have the 
well-thumbed look. If some have pages stuck together, so much 
the better. Paste a sticker in each one, and leave the books in 
places your target frequents, and even a few he doesn't. Some 
good possibilities are: 
Chapter Four 
Destroying Your Target's Reputation 
55 
e His workplace, to let his boss and fellow employees know 
the sort of trash your target reads. 
e A park near his home, so that parents of small children can 
see the sort of material this man leaves around, 
e On the sidewalk near his home, as if your target had 
"accidentally" dropped it there. 
e Supermarkets and shopping malls in the neighborhood. 
Someone finding such a book may return it when your tar- 
get's wife is home alone. Although the timing is out of your 
control, the more books you seed throughout the area, the better 
the odds of this happening. 
The Ghost Picture 
If your target leaves his camera within your reach, borrow it 
for a couple of hours. Use it to take a close-up picture of an ac- 
complice’s genitals, then put it back where you found it, He will 
wonder where the extra photo came from when the pictures re- 
turn from the photo-processor, and if his wife or girlfriend is 
the one who picks up the picture, she may get the shock of her 
life when she scans through the prints. In a small town, the 
chances of the photo processor or store clerk's knowing your 
target's family or friends are greatly increased. Lf he's a respect- 
able and "happily married” man, a close-up shot of female, or 
even male, genitals will cause a stir. You can spend many 
happy hours imagining the scene that will take place when his 
wife asks him to explain that frame! 
Likewise if your target's female, and married. Her husband 
will be very curious regarding the origin of that photo of a 
man’s genitals.  Gaslighting 
36 
Truth is the Best Defense 
One absolutely terrifying way of destroying your target's 
reputation, or at least causing him severe embarrassment, is to 
publicize any of his dirty linen that appears in public records. 
This technique depends on factors totally out of your control, 
but if events move your way, you can make your target ex- 
tremely uncomfortable. 
Public records can be a gold mine of information. They can 
provide the raw material for destroying your target's reputation, 
credibility, and even his livelihood. If you keep an eye on the 
legal announcements page of your local newspaper, you'll 
regularly see legal notices regarding divorce, bankruptcy, and 
civil actions. If you're really determined, you can make periodic 
trips to the county court house to check records of trials and 
cases filed. This needn't be an aimless "shotgun" search, as 
you'll often get indicators of what to seek out. A rumor that 
your target's marriage is on the rocks, for example, will let you 
focus on divorce filings. If your target's vehicle has been repos- 
sessed, check out bankruptcy filings and civil actions, in case a 
creditor is litigating for payment of a debt, 
Once you zero in on a court action, spend a few dollars for a 
transcript of the legal records, Make photocopies and mail them 
anonymously to those who will be interested in knowing, such 
as relatives, neighbors, and fellow employees. 
Divorce isn't a scandal these days, but your target will feel 
very uncomfortable when he discovers that his friends and fel- 
low workers have copies of the legal document, with all its de- 
tails. Sending his boss a copy of a suit to collect debt will sug- 
gest that your target's wages may be gamisheed, and this can 
produce some uncomfortable moments. 
This information will cause even more serious ripples if you 
send copies to all the companies which have issued your target 
Chapter Four 
Destroying Your Target's Reputation 
57 
credit cards, and all stores where you know he has credit ac- 
counts, 
The more you know about your target's lifestyle, the better 
you'll be able to direct your search purposefully. A ladies' man, 
for example, just might wind up on the receiving end of a pa- 
ternity suit. Sending copies of the paperwork to his fellow em- 
ployees, neighbors, and clergyman will cause your target some 
uncomfortable moments. 
One free-lance writer suffered when an anonymous malefac- 
tor unearthed a trial transcript in which he had served as an ex- 
pert witness. Although this writer presented himself as an ex- 
pert in law enforcement and the use of force, some years earlier 
he'd written for a sexually-oriented magazine, and the opposing 
attorney in this case asked him if he was also an expert on the 
female orgasm. The unknown malefactor made photocopies of 
the testimony and mailed them anonymously to other people in 
the field, causing embarrassment and loss of reputation to his 
target. 
This is a good place to emphasize that cultivating sources of 
information about your target can pay big dividends for the 
time involved. If you work in the personnel office, you can find 
all sorts of leads regarding your target's lifestyle and potential 
problems. If the company provides medical insurance, you 
might find a record of payment for an alcoholism treatment 
program. 
Heavy drinking in a locale with no public transport points 
the way to scrutinizing traffic court convictions. Publicizing a 
Driving Under the Influence (DUD conviction can have serious 
repercussions, If, for example, your target took two weeks’ 
"vacation" time recently, and you discover he secretly spent the 
time in jail serving a DUI sentence, letting his colleagues know 
where he spent his holiday will help your cause. Likewise if he 
spent time at an in-patient alcohol rehabilitation program.Gaslighting 
58 
Also cultivate personal sources. Make friends with your tar- 
get's enemies and business rivals, who will be attuned to col- 
lecting dirt about him. A few well-chosen words, and maybe a 
few drinks, will elicit leads to damaging information, This can 
be explosive if it includes a criminal record. 
A felony conviction, no matter how long ago, is always bad 
news, especially if his employment application form had a 
space for listing a criminal record and he left it blank. Finding a 
felony conviction will require knowing where your target lived 
in his younger years, but once you discover this, you can direct 
your search to court records in that area, 
Notes 
1. Mesa, AZ Tribune, October 6, 1993. 
Chapter Five 
Provoking Confrontations 
59 
\chapter{Provoking Confrontations}
Chapter Live 
PROVOKING 
CONFRONTATIONS 
By this stage, you'll see the multiplier effect working for 
you. Gaslighting, producing wear and tear on your target's 
nerves, degrades his effectiveness. Another effect will be to 
make him edgy, and even paranoid. He's more likely to become 
involved in conflicts with others, as he follows the vicious spi- 
ral down to his destruction. 
Protected Persons 
If your target is female, or a member of a protected minor- 
ity, you'll find it more difficult to carry out some of these tac- 
tics. If female, you cannot take advantage of anything sex-re- 
lated, such as physical weakness. Employers today are very 
concemed about anything that can be interpreted as "sexual 
harassment" in court, which means that any problems you cause 
a female target at work should preferably be related to job per- 
formance. 
Likewise for ethnic minorities. If your target is a member of 
a "black power" organization, you can't use this against him the 
way you would a Caucasian target's membership in the Ku 
Klux Klan, Exposing a minority target's incompetence on the 
job, however, will have several healthy effects:Gaslightin g 
60 
e It will justify the feeling that your target is holding his job 
only because of "quota hiring,” and not because of his abil- 
ity. 
° it will provoke resentment from fellow employees, which 
while not finding free expression, will kick back at your 
target in other ways. He'll be frozen out socially, and fellow 
employees won't offer him the help they might provide to 
others having a difficult time at work. 
e Incompetence, if properly documented, can stand up in 
court as grounds for termination. Documentation must be 
solid, though, because the employer will have to be very 
careful before terminating a minority employee. 
Creativ 
We've already examined some ways to take advantage of 
existing conflicts between your target and another person. Now 
let's explore how to create some: 
Both on the job and at home, your target is vulnerable to 
artificially induced confrontations with fellow employees, fam- 
ily members, and neighbors, The simplest way to set the stage 
for a confrontation is to report any derogatory comments your 
target makes about any fellow employee or neighbor to the 
subject of the discussion. If you have a witness, your target is in 
a weak position either to deny having made the comments, or to 
make a scene when he's outnumbered. An accomplice to back 
you up can be very helpful here, if you decide to become crea- 
tive. 
Let's note that some of the following tactics can backfire 
unless your target's self-esteem and standing with others are 
already very deteriorated. If you tamper with his e-mail, he'll 
know that he did not write a memo attributed to him if it's a 
total fabrication. On the other hand, if you make a subtle 
Chapter Five 
Provoking Confrontations 
61 
change in a memo he's sending, he might well think that he 
made an error. 
In many real life situations, there is more than one potential 
target. This is all to the good, because it helps avoid the moral 
dilemma of provoking a confrontation between your target and 
an innocent person. In the workplace, your secondary target 
might be the owner's idiot son or the snitch. If there are two 
company spies in your workplace, for example, you can 
contrive situations that put them at each other's throats, to other 
employees’ benefit. While the two snitches are fighting each 
other, they won't have time or energy to rat on other colleagues. 
The Fine Art of Tnstigation 
Sometimes you have to create the conflict from scratch. In 
other instances, you can take advantage of a pre-existing n- 
valry. For example, two people might be competing for a pro- 
motion or raise. This is a perfect beginning for you, and all you 
have to do is to make your target, or his rival, suspect that one 
is using unethical tactics against the other. What you've got 
working for you is the common knowledge that office politicing 
is a minefield, and rivals have used the dirtiest and most 
outrageous Lactics against each other. 
One way to pour gasoline on the flames of an office rivalry 
occurs if your office has linked computers, as in a local area 
network, and uses e-mail. Your target may be tempted to spy on 
his rival by sneaking a peek at his computer files. If you have 
access to your target's, or to his rival's, computer you can plant 
a file that will cause a serious problem. This takes the form of a 
nasty and inflammatory letter to the boss:  
Gaslighting 
62 
Dear Mr. Bigwig: 
1 know that you've lately become dissatisfied with 
Harry Target's job performance, and perhaps | can 
provide some enlightenment. Harry's been interviewing 
Jor other employment lately, as I've seen him bring in 
the Daily Rag open to the help-wanted ads, and this 
probably explains his taking off without explanation in 
the middle of the day. 
I've learned that the reason he left his last job was 
heavy drinking. To date, I haven't seen any sure evi- 
dence of that here, but Harry's appeared a little un- 
steady at times, and we both know his judement and at- 
tention to detail aren't the best. It might be a gaod idea 
to smell his breath. 
He's also been insecure about his continued em- 
ployment here, ever since he made that major error 
with the Gottrocks account, and had that argument with 
Gottrocks himself. He said to me that, if he ever had the 
opportunity, he'd "shoot that son-of-a-bitch like a dog." 
With this attitude, it's hard to see him remaining 
with this company, much less receiving a promotion, 
Sincerely, 
Charlie Goodwill 
In real life, it's unlikely that any employee would actually 
send such a memo to his boss, but to the person examining his 
computer files, this appears to be the draft of a letter Charlie is 
contemplating sending to Harry's supervisor. As such, it's very 
credible, because it depicts Charlie's inimical intent. To a target 
Chapter Five 
Provoking Confrontations 
63 
already quivering with paranoia, it will appear to be an immi- 
nent and grave threat. 
Note that this is also effective without e-mail, in the form 
of a file on an ordinary computer disk. If Harry scrutinizes it 
and makes a copy to study the contents at leisure, he'll have an 
eye-opener. Harry will just assume that his rival intends to print 
it out as an ordinary memo. 
The low-tech variant of this comes into play if the office 
doesn't have any computers, but your target or his rival regu- 
larly search each other's wastebaskets. You simply draft out 
your denunciatory letter, type it with a carbon copy, and put the 
copy in the wastebasket. 
You can make this work even if your target doesn't rum- 
mage through his rival's wastebasket. You crumple the carbon, 
then flatten it out and attach the following unsigned note with a 
paper-clip or staple: 
Hey, Harry, 
I thought this would interest you. 
This note and the attached copy will put somebody in deep 
doo-doo, Of course, the note is not in your handwriting, as 
you've had an accomplice write it. Preferably, the accomplice 
doesn't work for the same company and your target doesn't even 
know him. It would expose part of your plan if the target were 
to recognize the handwriting. 
Yet another way to instigate a conflict, even in a low-tech 
office where the most sophisticated piece of equipment is the 
telephone, is to let the office gossip "overhear" your end of a 
conversation. You dial your home number, the weather report, 
or time of day, and after a greeting and a few pleasantries, carry 
on a conversation something like this:    

Gaslighting 
bt 
"Yeah, Joe, | know, there's office politics here, too. 
Why just the other day one of the people, he's really a 
troublemaker, Harry Target, was saying that Chesley 
Goodman is a real jerk the way he botched up the 
Trimble account last month. Harry said if he'd been in 
charge, he could have done a much better job." 
The office gossip will almost inevitably repeat this, and 
when word gets back to Chesley Goodman, there will be an ac- 
count to settle. There's only a slight risk that this will come 
back to bite you, because it's impolitic for the office gossip to 
admit that he was eavesdropping on a telephone conversation. 
Another theme: 
"There might be something opening up here for you 
soon. Harry Target was telling me the other day that he 
knows Chesley Goodman's on his way out. I'll let you 
know when that happens." 
Yet another rumor you can start the same way: 
"| heard Harry Target's on his way out, They'll be 
canning him one of these weeks because he hasn't been 
producing." 
The effect of this can be serious if Harry Target has truly 
made several recent mistakes on the job and is, in fact, worried 
about his job security. The lame-duck employee tends to be a 
pariah in American corporate culture, and Harry may find his 
friends, if he has any, shying away from him. 
Creating a conflict from scratch is more difficult than ex- 
ploiting an existing one and requires good information, careful 
judgment, and a sense of timing. You must know both pro- 
tagonists well enough to make an informed judgment regarding 
what will set off fireworks between them. This may be an eth- 
Chapter Five 
Provoking Confrentations 
65 
nic or religious difference, or one of what is euphemistically 
called "lifestyle." Let's examine a few scenarios that cover vari- 
ous methods of operation to suit individual circumstances: 
Percy is the owner's limp-wristed nephew, just out of 
college and working his first job with his uncle's com- 
pany for twice the salary experienced hands are get- 
ting. Harry "Macho" Target is the shipping foreman, 
and you whisper to Percy that Harry hates queers, and 
that he'd better stay out of Harry's way because of his 
violent nature. You know that Percy makes up the pay- 
roll and covers other disbursements, and take advan- 
tage of your position in the accounting department to 
make one of Harry's expense chits er overtime slips 
disappear, When Harry comes to you to chase down his 
payment, you send him to Percy with this piece of ad- 
vice: 
"Just sweet-talk Percy and you'll get anything you 
want from him. He told me he thinks you're a real 
hunk." 
Even one such encounter can make the sparks fly. 
Homosexuality isn't the only issue that you can manipulate 
to provoke a confrontation. Let's look at a couple of ethnic en- 
tanglements: 
Irving is the company snitch, and Harry Target is a 
noted brown-noser. You mention to Irving that Harry 
hates Jews with a passion, and if this doesn't produce 
immediate results, you place a bumper sticker on Irv- 
ing’s car. The sticker reads; “Hitler Was Right!" Then 
watch the fur fly. If Irving doesn't notice the sticker at 
Jirst, you point it out to Harry, as if you're just showing 
him a joke. Next day, you scrape the bumper stickerT Gaslighting Chapter Five 
| Provoking Confrontations 
66 67 
from Irving's car and stick it on Harry's bumper, and Note that it doesn't have to be Sturdley's car; it can just as 
run an old-fashioned "church key" can opener down well be his basement window, or even his porch or living room 
the side of his car several times to get his attention. window. It also doesn't matter if Sturdley discovers the hose 
= j before the torrent of water causes a flood, You can't measure The protagonists don't have to be fellow employees, al- AS ‘ though company politics often helps to season a conflict. You the success of your instigation by raw physical damage. In other 
can create problems between neighbors, as well: words, it's the thought that counts. 
Harry Target is a Vietnam veteran, and like many 
Vietnam vets, hates Jane Fonda. You or an accomplice 
mention to him that Rick Sturdley, his neighbor, ad- 
mires Jane Fonda both for her films and for her politi- 
cal courage in going behind enemy lines during the 
war. A couple of days later, you notice Harry's garden 
hose on his front lawn, and when you have the oppor- 
tunity to do so unseen, you move it to water Sturdley's 
lawn. Go home, pull a chair up to the window, and 
watch what happens when Harry comes home. 
You don't have to let it go at that. The odds are that the first 
incident will result only in an argument, but you can provoke 
follow-up incidents: 
Harry cames home and sees his garden hose water- 
ing his neighbor's yard from his faucet. Turning it off. 
he angrily rings Sturdley's bell, and informs him that 
the next time Sturdley uses his hose and his water, he'll 
get a dousing of water where it will do the most good. 
He rejects Sturdley's denials, and repeats his threat. 
Late that night, you slip the end of Harry's hose inside 
a crack in Sturdley's car window, taking the usual care 
not to be seen. Set your alarm so that you wake up in 
time to watch the fireworks when Sturdley discovers his 
water-logged car. 
It helps to be very indirect when you create a situation from 
scratch, Use others to do some of your work for you. Even to- 
tally imaginary characters will do: 
Pedro lives next to Harry Target. One night, you run 
the blade of a screwdriver or “church key" down the 
side of Harry's car several times, Next day, you com- 
miserate with Harry when he wails over his ruined 
paint job, and mention that the same thing happened to 
your cousin the previous week. You add that your 
cousin had seen several Hispanics running from the 
scene. Your wife or accomplice mentions to Pedro's 
wife that Harry suspects that Hispanics put the 
seratches in his paint job, and that he hates and sus- 
pects Hispanics because he feels that they're dirty, 
sneaky people. That night, after making sure that every- 
one else is asleep, you put scratches on Pedro's vehicle. 
Just in case that doesn't get things going, the following 
night you slash Harry's tires. 
Exploiting Political Correctness 
Some companies have definite and identifiable corporate 
cultures, and employees who want to get along must conform. 
Non-conformity can result in being frozen out of pay increases, 
promotions, and bonuses. In some cases, non-conformity is a 
violation of company rules. Let's look at a few examples:——— 
 
Gaslighting 
68 
Some companies have "no-smoking" rules, both on and off- 
duty, although some states have now passed laws that forbid an 
employer from banning off-duty tobacco use. Still, if the corpo- 
rate culture is anti-tobacco, a smoker or chewer will find him- 
self frozen out socially, and even left out of bonuses and pro- 
motions, 
Exploit this by planting a pack of cigarettes in one drawer 
of your target's desk. Another way is to put a cheap pocket bu- 
tane lighter on top of his desk. Leave it to him to explain it's not 
his, especially when you plant another lighter on his desk 
whenever he's not looking. Yet another way is to place a pack 
of cigarettes on his car seat, where it may be noticed by fellow 
employees and supervisors. The best way, of course, is if you 
can obtain a jacket exactly like your target's, and put it in a 
room or closet full of tobacco smoke for a day or two. Bring it 
to work in a plastic bag so it doesn't air out, and substitute it for 
your target's jacket when you have the opportunity. With the 
jacket reeking of tobacco, you won't have to depend on some- 
one "finding" a planted pack. If the company has a group health 
plan with different premiums for smokers and non-smokers, re- 
percussion’ can be more serious. 
Alcohol on the job is bad news, both because employers are 
concerned about deterioration of performance by employees 
with alcohol-fogged brains, and because of liability if the em- 
ployee gets injured on the job. Planting a small bottle of liquor 
in your target's desk will raise eyebrows if anyone sees it. If 
there's an employee coat rack, one way to ensure that the word 
gets around is to buy small “airline” bottles of whiskey, and 
plant one in your target's coat pocket. However, first loosen the 
cap to ensure it leaks. Soon, the odor will be conspicuous, and 
will lead right to your target's coat, with the incriminating evi- 
dence right in the pocket. 
Chapter Five 
Provoking Confrontations 
69 
Remember that exploiting political correctness by planting 
items on your target's desk can work for only a few times before 
he realizes that the incriminating items didn't get there by them- 
selves. Planting material too early in the game can alert him to 
the danger, which illustrates that proper timing is everything, 
Introduced at the right moment, planted material can be very 
effective, because others will not believe that your target is a 
victim of foul play. 
Attitudes and beliefs are not illegal, nor are they grounds for 
termination, under the law. However, there's a world of differ- 
ence between statutes and the "unwritten law," and an employee 
who expresses, by word or action, attitudes foreign to the cor- 
porate culture can measure his future career with a stopwatch. 
This is the key to putting your target in the hot seat. There's an 
array of tactics that can make it appear that your target em- 
braces attitudes offensive to the corporate powers-that-be. 
While feminists claim that American business has a "glass 
ceiling” to keep women down, some female-owned companies 
have "lace curtains,” which impose their own demands. Any 
male employee of a "lace curtain" enterprise has to watch his 
step. A Playboy or Penthouse magazine on your target's desk, 
for example, is evidence of the worst crime of all, "male chau- 
vinism”" that "exploits women as sex objects.” 
A bumper sticker with a slogan that's politically incorrect 
for that company's culture can be as effective. "Gay and Proud," 
and "Honk If You Love Jesus" are two strong statements guar- 
anteed to offend somebody. So is a copy of Adolf Hitler's Mein 
Kampf. Fortunately, there's a wide variety of bumper stickers 
available at low cost. Caution: Don't use the bumper sticker 
tactic until your target's already being flushed down the toilet.  
Gaslighting 
70 
Confidential Papers 
One good way to provoke a confrontation with an employer 
is to remove some confidential papers from the supervisor's of- 
fice and place them on your target's desk. A day when your tar- 
get is absent provides the best opportunity for this. 
The next tusk is to let the boss know of the unauthorized 
possession. You might get a chance to tell him directly if the 
supervisor comments in your presence that he can't find the 
XYZ Report. You then say to him, "You must have lent it to 
Harry Target. He's got it on his desk." Unfortunately, this tactic 
depends on chance. 
Another and more direct way lies open to you if you have 
legitimate reason for access to a particular confidential docu- 
ment. You say to your boss: "/'d like to copy some figures from 
fast month's production report. Can I borrow it fram Harry's 
desk for a few minutes?" 
Another way is to have an accomplice tell the boss. This 
isolates you from the event. Yet another way is to pass the work 
to the office ass-kisser or company snitch, This has a three-fold 
effect. First, it gets your target in trouble. Next, it exposes the 
snitch to resentment and possible reprisal from your target if he 
finds out who snitched to the boss. Finally, it can provoke a 
confrontation between the target and the snitch, or even an un- 
involved person, 
If your target is slightly paranoid, he may suspect someone 
who had nothing at all to do with the affair of denouncing him 
to the boss, You can help this happen by mentioning 
"confidentially" to your target that Joe Brown-nose had a long 
discussion with the boss the day before. 
Chapter Five 
Provoking Confrontations 
71 
Personal Property 
It may be difficult to obtain a supervisor's confidential re- 
ports, especially if he keeps his office door locked or keeps the 
papers in a safe when not in use. However, you don't absolutely 
need access to confidential documents to put your target in the 
soup with his boss, because you can take advantage of the fact 
that many people strongly resent others borrowing their prop- 
erty without permission. 
Next time you're in your supervisor's office, slip his desk 
lighter into your pocket when he's not looking. If he doesn't 
smoke, there are other personal items you can “borrow,” such as 
a monogrammed pen, gold-plated letter-opener, etc. After you 
leave, drop the purloined item on your target's desk. 
Exploiting the Snoop 
Although it's incorrect to mention the topic in polite society, 
many offices have snoops and gossips. Some snoops go to the 
extreme of rummaging through fellow employees’ desks or trash 
to find out what they can. This offers the perfect opportunity to 
start a rumor behind your target's back while remaining overtly 
aloof. A short list of items that can provoke malicious gossip 
follows: 
e An employment application for another company, partly 
filled out and crumpled. 
e A "draft copy" or "carbon copy" of your target's resume, re- 
cently updated. 
e A publication going against corporate culture, such as a 
copy of The Daily Worker, or a copy of Playboy in a femi- 
nist-dominated business. 
When exploiting the person addicted to gossip, it helps to 
have one or more accomplices. One of your accomplices can 
reinforce the "rumor" by stating it again to the gossip. Another    
Gasl ighting 
72 
can keep his ears open and report back to you if the rumor you 
instigated is circulating around the workplace. 
The office gossip can become your inadvertent ally if you 
handle the situation deftly. The gossip thrives on rumors, and 
you can use this channel to spread damaging information and 
even "disinformation" anonymously. 
Disinformation plants erroneous ideas in the target's mind, 
to mislead him and cause him to make serious errors. A pro- 
gram of disinformation can lead the target to making serious 
trouble for himself, and in a way that can't be traced back to 
you because he did it all to himself, 
Let's scrutinize a few examples: 
Harry Target and Joe McNasty are rivals for an im- 
pending promotion. You tell the office gossip “and keep 
this to yourself” that Harry was bad-mouthing Joe to 
the boss last week. 
Sometimes, a rumor alone can provoke a conflict. In other 
cases, you have to help it along. The opportunities are many: 
You begin as before, telling the office gossip in con- 
fidence that Harry was carrying bad tales about his 
rival Joe to the boss, Two days later, your accomplice 
mentions that Joe is pretty burned off about Harry's 
unethical tactics. Two days after that, Harry finds his 
tires slashed. 
The most effective way to exploit office gossip is to tailor it 
to fit existing occurrences: 
Harry and Joe are bitter rivals. One day, upon 
leaving the office, Harry finds that his car's been side- 
swiped in the company parking lot. You whisper in the 
gossipy person's ear that maybe, just maybe, a certain 
Chapter Five 
Provoking Confrontations 
73 
person might have had it in for Harry, Give your "spin" 
on the situation a day or two to circulate, then put some 
deep scratches on Joe's car with a screwdriver. 
You can even use the gossip to sabotage your target by 
proxy and posthumously, so to speak. If you're leaving the 
company, and you know that your target's been hungering for 
your job, you can give your target a painful mental hot foot by 
telling the gossip that your boss confided in you that Harry 
would have the position as soon as your desk was vacant. You 
can only imagine Harry's poignant disappointment when the job 
goes to someone else, and he later discovers that he wasn't even 
in the running. Another example can take place if Harry's the 
one leaving: 
Harry Target has found himself a better-paying job 
elsewhere, and you're familiar with the company. You 
know Mr. Savage, the plant manager, to be a very up- 
tight and religious person who hates to hear profanity. 
You spread the disinformation through your ally the 
office gossip that the way to get along with Mr. Savage 
is ro tell him a dirty joke every morning. 
You can even induce your target to sabotage his chances of 
getting a certain job by adroitly devised disinformation: 
The gossip tells you that Harry Target is testing for 
employment with XYZ Company, and you know that one 
of the tests they administer is the Rorschach ink blot 
test. You feed back to the gossip that your cousin, a 
psychologist, had told you that the way to pass such a 
test is to provide a lot of violent responses, such as 
Seeing guns, atomic explosions, dead animals, and 
other blood and guts responses to the blots. You say 
that businesses today seek aggressive personalities be-  Gaslighting 
74 
cause they're go-getters, and that a pattern of aggres- 
sive responses will almost guarantee getting the job. 
This is precisely the wrong information, because a pre- 
ponderance of blood and guts responses on the ink blot 
lest presents a picture of psychosis. However, if Harry 
accepts this disinformation at face value and acts on it, 
he'll screw himself right out of the job. Furthermore, 
he'll have compromised his chances of ever being em- 
ployed by that company, no matter how many times he 
applies, because test results often remain in personnel 
files for many years. 
Yet another instance of disinformation depends on your 
acting upon several items of correct information, and may be 
hard for you to carry out, It's worth examining, though, because 
if you ever get the opportunity, you can deliver a devastating 
strike against your target: 
You know that Harry's been seeking other employ- 
ment, and you even know several companies where he's 
applied. You have an accomplice telephone Harry to 
say; 
"This is Mr. Savage of the XYZ Company. I'm glad 
to tell you that we selected you for the opening. When 
can you begin?" 
Believing that the job is his, Harry gives notice, and 
two weeks later, he's high and dry. 
Note that some companies have become notorious for im- 
mediately terminating any employee who gives notice. Instead 
of letting him work out his two weeks, they have him escorted 
to the gate at once. The theory behind this is that the employee 
won't be giving his best with one eye on the door, and they're 
Chapter Five 
Proyoking Confrontations 
75 
better off without his dubious services. Another rationale is that 
the departing employee will be packing proprietary information 
in his attaché case to take to his new job, and immediate termi- 
nation reduces the risk of industrial espionage. Either way, Mr. 
Harry Target is out on the street, PDQ. 
Door Slamming 
A basic principle when leaving a company is to go grace- 
fully, and "not to slam the door,” because you may need their 
favorable reference one day. One deft way to induce your target 
to slam the door, and hard, begins with the same tactic: 
Harry gets a call that he's got the job he'd been 
seeking. You know that there's been tension between 
Harry and Mr. Hardnose, his immediate superior. After 
Harry's given notice, he takes a couple of days’ sick 
time. At this point, you tell the office gossip that you 
regularly pass the company that hired Harry on your 
way to work, and that this morning you noticed a car 
that looks very much like Harry's in their parking lot. 
If Harry doesn't take any time off, you have to try another 
tactic: 
After Harry's given notice, you or an accomplice ask 
him during lunch or break-time how he feels about get- 
ting out from under Mr. Hardnose's thumb now that 
he’s leaving. Unless Harry's very restrained, those pre- 
sent will get an earful. If the office gossip wasn't on the 
scene, you can carry the news to him, and let it trickle 
down to Mr. Hardnose. If Mr. Hardnose is the least bit 
sensitive, sit back and enjoy the fireworks. 
Even after Harry leaves, you can torpedo his chances of 
obtaining a favorable reference in the future, and totally close  Gaslighting 
76 
out any chance of his returning to that company. You and an 
accomplice discretly leak to the boss that Harry had had many 
unkind things to say about him while employed there. 
Direct Action 
This tactic requires one of two things: 
e You're in very good standing with the boss, so that he'll ac- 
cept your word without question, or: 
e You have several accomplices within the company who 
work in conjunction with you. 
Recruiting accomplices isn't hard if your target really de- 
serves what he gets. If your target is a nasty, unpleasant, or 
treacherous person, he'll have created a backlog of unsettled 
scores. Fellow employees will almost be standing in line to have 
a go at him. 
If you're working alone, wait until your target calls in sick. 
Go to the boss the next day and tell him that your wife saw your 
target at a shopping mall. 
Alternatively, you can tell the boss that the target's wife 
called your wife and said her husband didn't make it to work 
because of a severe hang-over. 
If you have partners in crime, you can be more exploitative. 
When there's an important meeting scheduled, one of your ac- 
complices tells your target that the meeting's been canceled, in 
the presence of another in on the plot, When the meeting takes 
place, if the boss notices your target's absence, you sidle up to 
him after the meeting and say: 
"Look, I may be speaking out of turn, but Harry told me he's 
sick and tired of attending these meetings. He said nothing gets 
done, anyway." 
Chapter Five 
Provoking Confrontations 
W 
If the boss confronts the target, and the target replies he was 
told the meeting was canceled, there are two people who will 
testify that he was told no such thing. 
e-mail 
Working with your target produces many opportunities to 
sabotage his life. If you know your target's schedule, especially 
his roster of appointments, you can go on a rampage. E-mail is 
easy to forge when you want to embarrass the target, and even 
provoke confrontations. Unless there's an unbreakable security 
system in your office's e-mail, you can deposit messages 
"signed" by your target, or alter a genuine message written by 
him. 
An example is a faked message canceling an important 
business meeting: 
H. E. Gottracks 
Gattrocks Construction Co. 
Dear Mr. Gottrocks: 
I regret to inform you that I have to go out ef town 
on Monday, the 18th, and cannot, therefore, meet you 
at your office to finalize the terms of the new contract. 
I'll have my secretary call you for a new appointment, 
Sincerely, 
Harry Target 
Another way is to change the time or place of the meeting: 
H. E, Gottrocks 
Gottrocks Construction Co. 
Dear Mr. Gottrocks:  Gaslighting 
78 
I would like to change the date of our meeting, as I 
have to go out of town on Monday, the 18th, and can- 
not, therefore, meet you at the Greasy Spoon restaurant 
to discuss the terms of the new contract. I'd like to meet 
you the following day, same time, at the Hung Low 
Chinese restaurant on 18th Street, and hope that this 
will be convenient for you. If not, please advise me. 
Otherwise, I'll see you on Tuesday the 19th, 
Sincerely, 
Harry Target 
If the meeting was originally to take place at a restaurant, 
the target arrives but Mr. Gottrocks doesn't. If the meeting is 
scheduled at Mr. Gottrocks' office, your target shows up but 
Mr. Gottrocks isn't expecting him. 
At this point, the multiplier effect may kick in to enhance 
the disruptive power of the gaslighting. If Harry Target is an- 
noyed with Mr. Gottrocks for missing the appointment, he may 
confront him indignantly. Mr. Gottrocks, in turn, will wonder 
why Harry Target is blowing his top, as he’s the one who 
changed the appointment. He may show the message changing 
the appointment to Mr. Target, who will deny he ever sent it. 
This may not sound acceptable to Mr. Gottrocks, who may 
conclude that Mr. Target simply forgot. If the argument be- 
comes heated, the following consequences may ensue: 
Mr. Target alienates one of his company's clients. 
Mr, Gottrocks complains to Mr. Target's boss about his be- 
havior. 
Rumors begin to circulate that Mr. Target is losing his grip. 
Chapter Five 
Provoking Confrontations 
79 
Sexual Harassment 
While you have to watch your step if your target is female, 
the situation can turn if you handle it adroitly. There are several 
ways to put your target on the hot seat, facing accusations of 
sexual harassment. If your target is male, the basic tactic is to 
send notes and gifts to a female fellow employee. In some 
cases, your target will be your unwilling accomplice, if he has a 
reputation as a “ladies’ man." Even without a reputation, if he's 
ever made a comment to another employee that he finds a 
woman attractive, his words will return to haunt him, and make 
his denials unbelievable. 
Another use of e-mail is to write a romantic note to a female 
employee, signing your target's name. If the note is fairly 
explicit in its sexual language, and the woman is married, it can 
cause a stir, However, you don't need this combination of cir- 
cumstances to make this tactic work. 
With "sexual harassment" a hot and emotional topic today, 
simply sending romantic notes to a woman who doesn't appre- 
ciate receiving them will bring a charge of sexual harassment. 
Such an emotionally-laden accusation makes your target 
"guilty" until proven innocent, The effect will be explosive if 
the company is owed by a female, or staffed by a group of 
“feminists.” 
It's worth remembering that this tactic will work even if 
your office doesn't use e-mail. Simply typing a suitable note on 
your target's typewriter one day when he's absent will produce 
the same effect. Sending it in the inter-office mail, or simply 
dropping the note on the woman's desk, serves perfectly for de- 
livery. 
You can compound the effect of the trumped-up sexual har- 
assment by making some anonymous phone calls. Once you 
deliver a note or two in your target's name to a female em-  Gaslighting 
80 
ployee, begin calling her late at night, and breathing heavily 
into the phone. If the female recipient of the notes brings 
charges, it's even worse for your target. You continue the heavy 
breathing during late-night calls, and the obvious conclusion is 
that your target is the guilty party. 
Nasty, Nasty 
Some workplaces have no females. A variation on the sex- 
ual harassment tactic becomes necessary, and one is the homo- 
sexual note, in which your target makes an adyance to a male 
staff member: 
Dear Charlie: 
Since you've come on staff, I've been impressed by 
haw well you do your work, and by your charm and 
personality. I feel we should get to know each other 
better. How about dinner together at XYZ Restaurant 
tomorrow evening? 
Anxiously awaiting your reply, 
Harry Target 
To make sure that there's no misunderstanding, the "XYZ 
Restaurant” mentioned in the note is a well-known gay bar. 
Variations 
Setting up your target for a sexual harassment charge is a 
very powerful tactic because it's workable in a number of ways, 
using high-tech or low-tech methods. You'll have the employer 
on your side, because no employer wants to face a lawsuit for 
allowing or condoning sexual harassment in his workplace. You 
don't necessarily have to write a series of notes. You don't have 
Chapter Five 
Provoking Confrontations 
81 
to have access to e-mail. All you have to do is have flowers or 
candy delivered to a suitable subject. 
Some florists deliver, and you have an accomplice go to the 
florist to place the order directly, paying in cash, Using the 
telephone to place the order, and having it billed to your target's 
phone bill, doesn't work as well these days, especially if your 
area has "Caller ID." 
Florists provide gift cards with their floral arrangements, 
and your accomplice simply signs your target's name on the 
card. It doesn't matter if the signature is not an exact match, be- 
cause it's extremely unlikely that the signature will ever be ex- 
amined by a handwriting expert. The odds are great that nobody 
will believe his denials, even if the handwriting doesn't look 
like his. 
The first bouquet has a signed card. After the female em- 
ployee expresses her resentment, another bouquet arrives a few 
days later, with an unsigned card, The following week, a box of 
candy arrives via UPS, and next week brings a piece of inex- 
pensive jewelry, Each present provokes a nasty response from 
the woman, or even her supervisor, if she’s brought this to his 
attention. At this point, it doesn't matter if your target has de- 
nied ever sending the first bouquet, he'll have the can pinned 
onto him regardless. A few heavy-breathing phone calls can't 
hurt. In fact, the telephone is useful for many other tactics. 
Telephone Games 
E-mail isn't the only vehicle you can use. If you have a fe- 
male confederate, she poses as a business contact’s secretary, 
and telephones your target or his secretary, if he has one, to ar- 
range or cancel appointments. The effects can be explosive in 
some cases. If an important client is scheduled to appear at a 
certain time, and your accomplice telephones to “cancel” the  Gaslighting 
82 
appointment, your target may be out of the office when the cli- 
ent shows up, which will cause an affront. 
Citizen's Band radio provides a perfect opportunity to pro- 
voke a confrontation between your target and a very aggressive 
person. You simply start an argument with someone on the CB, 
insult him, and state that if he wants to do something about it, 
you're prepared to face him down at any hour, day or night. If 
the person you're insulting is a nasty biker type, or worse, your 
target will be in real trouble. 
When using this tactic, don't worry too much about involv- 
ing an innocent person in your nasty little game. Remember, 
decent people don't accept challenges to fight from an idiot on 
the CB. 
The Missing Report 
and Other Choices 
An office rival will sometimes try to sabotage an adversary's 
work by swiping a crucial report from his desk, but this trick is 
too well-known to work very well today. The only hope you 
have is if you know that your target has a rival, and will blame 
the report's absence on him. 
One reason swiping a report doesn't work well is that most 
offices have computers, and the employees do their work on 
disks, producing a print-out only when they need a hard copy to 
submit to the boss. Therefore, swiping the paperwork won't do 
much damage, especially if your target has a high-speed printer 
that can produce another copy in minutes. 
Instead, you can destroy the report's effectiveness, and make 
it a liability for your target, by altering some facts and figures in 
Chapter Five 
Provoking Confrontations 
83 
it, There are several ways to embarrass your target, and earn 
him a reprimand from his supervisor, by taking advantage of 
moder computers. 
The first is to gain access to your target's computer and 
make the alterations directly. However, this may not be possible 
if your target has a security system on his computer, and has to 
enter a password to gain access to the files. Unless you know 
the password, you'll be locked out of access. 
If you're a computer hobbyist, and have talent as a "hacker," 
you may devise a way to gain access without the password. Al- 
temnatively, you may discoyer the password, taking advantage 
of the fact that most people use their middle names, addresses, 
names of family members, or street addresses as passwords, be- 
cause they're easy to remember, In fact, one computer expert 
listed a group of passwords that he stated are used 90 percent of 
the time.! Look up the list and use it as a starting point for 
breaking into your target's computer. 
If your target's computer isn’t a "stand-alone" unit, but is 
linked into other machines in the office with a local-area net- 
work (LAN), you may be able to gain access to his files from 
your keyboard. This allows you to "down-load” his report and 
make suitable alterations. Then you can print it out, and substi- 
tute it for the report your target submits. 
Workplace Sabotage 
Sabotaging your target's work is an excellent way of pro- 
voking a confrontation with his supervisor, but it's effective 
only if it appears to be the result of his carelessness. This is why 
setting fire to his work station is counter-productive. The same 
goes for deleting massive amounts of data from his computer. 
There are other and more subtle ways of sabotaging your tar- 
get's work to make it appear to be entirely his fault:  Gaslighting 
84 
In a machine shop, exploit any break your target takes to 
alter the settings on his machine after he's set it up and 
passed first-piece inspection. Turning a knob, or changing 
the computer-control setting, can change a critical dimen- 
sion by a couple of thousandths of an inch, enough to tum 
an entire lot of parts into scrap metal. Another way is to re- 
move one part from the lot, making it appear that he spoiled 
it and discarded it to hide his mistake. 
Another tactic can have even more serious repercussions, 
because it involves another employee, and is only suitable if 
you have a reason to “get” both of them. Remove your tar- 
get's micrometer or caliper and put it at the other's work 
station, or vice versa. This will inevitably result in an accu- 
sation of taking the instrument without permission, and if 
one or the other is hot-tempered, can produce fireworks. 
A delayed-action tactic along the same lines is to re-cali- 
brate your target's micrometer or caliper, setting it to read 
one or two thousandths high or low. If your target's a con- 
scientious machinist, this may not work well because he'll 
check his micrometer each moming with a "Jo Block," a 
precision-made block of stable metal used to calibrate in- 
struments. The tactic you use then is to gimmick his Jo 
Block. Borrow it long enough to grind a couple of thou- 
sandths off one side, then polish it to a mirror finish to 
match the original surface. When he next uses it to check 
his calibration, he'll inadvertently set off the calibration of 
any instrument he checks with it. 
A quick and dirty way to produce the same effect can work 
if your target has only one calibration block in his toolbox. 
Simply substitute another of a slightly different dimension. 
It's very unlikely that he'll scrutinize the engraved number 
on it every time he uses it to calibrate his instruments. 
Chapter Five 
Provoking Confrontations 
85 
In a retail outlet, make an extra entry on his cash register, 
He'll come up inexplicably short in his tally at the end of his 
shift. 
Another way to alter his tally is to remove a refund slip 
from his drawer. If he's having a busy day, he may not even 
remember how many refund slips he put in, and at the end 
of the shift will be unable to account for the shortage. 
If you work in a service industry, swipe one work order 
from the stack on his,desk or on his clipboard, to make it 
look as if he had lost it. He'll have to explain why the repair 
or work order wasn't performed. 
If you and your target work for the postal service, swipe 
letters from your target's bundle and drop them into an alley 
somewhere on his route. You may have to do this several 
times before a citizen reports the discarded mail to the post 
office, but when the ship hits the sand, it will hit hard! 
There are severe penalties for misdirection of mail. How- 
ever, be prepared, in some locales, for some very apathetic 
citizens who wouldn't report it if their neighbor's house 
were on fire. In such a case, make an anonymous call to the 
postal inspectors. 
If your target is a photographer, open the back of his camera 
momentarily while he's not looking. You'll fog his film and 
he'll have several totally spoiled exposures to explain away. 
If you want to be more subtle, open his camera in subdued 
light, to produce a light fog. This will not obliterate the im- 
ages, but will make them very difficult to print. Note that 
this may not work if he uses a modern electronic camera 
that automatically resets the exposure counter to zero when 
the back is opened. The counter will betray that something 
is terribly wrong. 
In a photo processing plant, surreptitiously add acid, such 
as sodium sulfite, to your target's developer. This is particu-  Gaslighting 
86 
larly effective with a film or paper processing machine, be- 
cause it will degrade the quality without making it too obvi- 
ous that something's gone wrong. Add very little acid to the 
replenisher, and the effects will be slow in coming. Worse, 
as he adds more replenisher to compensate for decreased 
uctivity of the developer, he'll only aggravate the problem. 
Only a chemical analysis, highly unlikely under the cir- 
cumstances, will disclose this form of sabotage. 
In a custom photo processing laboratory, go into your tar- 
get's darkroom during his lunch break and flash several 
boxes of printing paper. Don't make it too obvious. Expose 
the paper just enough so that it will turn a very light gray. 
If you both work in a high-security plant that requires an 
employee badge to enter and to leave, swipe his badge just 
before quitting time, putting it in a place where he logically 
could have dropped it. If you do this on a day you know he 
has to be home early, he'll be frantic while trying to find it. 
Some high-security companies have access controlled by 
"card keys," magnetic-striped cards that open locks and 
gates. If you have even momentary access to your target's 
card-key, wipe a powerful pocket magnet over the stripe, 
the same technique that negates ATM cards. Your target's 
card will suddenly become inoperative, with annoying re- 
sults. 
In a hospital, opportunities for workplace sabotage are end- 
less, but it's necessary to use fine judgment to avoid any 
harm to innocent persons. One harmless way to indict a 
hospital employee responsible for paperwork is to remove 
the consent for surgery or treatment from a patient's chart 
after the treatment has taken place, For legal reasons, con- 
sent slips are essential in hospital practice, and the absence 
of a consent slip after major surgery causes severe concern. 
Chapter Five 
Provoking Confrontations 
87 
e If you don't have time to rummage through a chart to re- 
move a couple of papers, make the whole thing disappear. 
Losing a chart is a major black mark for any nurse, 
e <A charge nurse with the keys to the narcotics cabinet has 
full responsibility for its contents. If the opportunity arises, 
removing some of the contents and disposing of them in the 
garbage can cause a major investigation, with inevitable re- 
percussions against the nurse. Just make sure you appear 
totally uninvolved. 
e Fast-food outlets offer several opportunities for sabotage. If 
your target cooks hamburgers, leave a box of frozen patties 
out next to his grill. The supervisor will see them and ad- 
monish him that taking patties from the freezer long before 
cooking makes them spoil. 
The Job-Seeker 
Many employers take an uncharitable view of a staffer’s 
seeking other employment, viewing this as blatant disloyalty. 
As we've seen, a few interpret it as such a serious affront that 
they'll fire summarily any employee who gives notice. In any 
case, an employer considering one of his employees for promo- 
tion will reconsider upon learning that the employee is job- 
hunting. Take advantage of this by “framing” your target as a 
job-hunter. 
The simplest way to do this is to have an accomplice, un- 
known to the target's employer, telephone his supervisor: 
Good morning, I'm Jack Frost, personnel manager 
at the XYZ Company, and I'm calling to verify Mr. 
Harry Target's employment with your firm. 
Another way to do this is to take advantage of the situation 
if your target's mail is opened by his secretary or in the mail 
room. You use a purloined letterhead from another company to  Gaaliighting 
88 
write a letter thanking your target for appearing for an employ- 
ment interview, and telling him that he'll be notified shortly 
whether he has the job. Office gossip being what it is, it’s al- 
most inevitable that the news will spread. The reason? 
If his secretary doesn't like him, she'll be happy to spread 
news that will hurt him. It may be even worse if his secretary 
does like him, and has a strong feeling of loyalty towards him. 
In such a case, she’s bound to feel hurt that her boss didn't con- 
fide in her, and her feeling of loyalty can easily turn to loathing. 
What if your target doesn't have a secretary? What if he 
opens all of his own mail? You can still make this work, using 
at least two other methods, 
The first is to type the same letter, and mail it to yourself at 
home, addressing the company envelope in pencil. This is to get 
a canceled stamp on the envelope. Upon receiving it, you open 
the envelope, erase your name and address, and type the name 
and home address of your target. You then put the letter back in 
the open envelope, throw it on the floor, and step on it a few 
times to scuff it. 
The next problem is getting someone in your company to 
read it, Some people who pick up open envelopes are tempted 
to read the letter inside. You can test for this by bringing an in- 
nocuous letter addressed to yourself to work and dropping it on 
the floor, with a hair folded over the letter inside the envelope. 
When someone returns it to you, check to see if the hair is still 
in place. If not, Bingo! That person has read your letter, and it's 
reasonable to suppose that he will read any other letter he finds. 
It's especially helpful if this person is your target's rival or su- 
pervisor. Next morning, take the forged letter in to work with 
you and drop it where the same person is likely to find it. 
Another way to get the forged letter in front of the supervi- 
sor's eyes is to photocopy it, and send it anonymously through 
the office mail, or simply drop it on the supervisor's desk. You 
Chapter Five 
Provoking Confrontations 
89 
can kill two birds with one stone if your target's rival has 
printed memo pads saying, "From the Desk of Johnny Rival." 
You swipe a few sheets, and have your accomplice, whose 
handwriting is unknown at your company, print the supervisor's 
name on one sheet, then staple it to the forged photocopy, 
Yet another way to put your target neck-deep in the soup is 
to fill out an employment application for another company in 
his name, if you can forge his writing or printing reasonably 
well, and leave it on his desk one day during his absence. An 
alternative tactic is to photocopy it and send the copy anony- 
mously to the boss, with an unsigned note saying: 
"Look what Harry Target had on his desk the other day, ” 
Don't forget to mail out résumés in your target's name to a 
number of companies, especially those where the owner or one 
of the executives knows your target's supervisor. Word will leak 
back. 
The fake job search can also sabotage your target's home 
life. Letters from companies arriving at home can provoke a 
lively discussion between your target and his wife, who will 
demand to know why he hasn't confided in her that he's job- 
hunting. 
If by chance you discover that your target is really looking 
for another job, this will provide the opportunity to hit him with 
a series of sinister gut-punches, to torpedo his chances of being 
hired. As pointed out in the first chapter, the more you know 
about your target, his aspirations and interests, the better your 
chances of carrying out an effective gaslighting campaign. 
Put together a list of the most likely job prospects for him. 
Include jobs advertised in the newspapers, but don't forget 
companies in a similar line of work. Prepare an outrageously 
fictitious résumé that's sure to attract negative attention. One  Gaslighting 
90 
way to sabotage his prospects is to list a totally fictitious last 
job. When the company's personnel department checks his 
background, the last job listed is the one getting first attention. 
Another point is to list phony educational attainments. Let's 
look at the way you might do it for a target seeking employment 
as a shipping clerk: 
RESUME 
Harry Target 309 Jones Street 
Punkintown, USA 
Age: 39 
Married, two children 
Health: good, 
EMPLOYMENT HISTORY: 
1989-1994: Purchasing Agent for General Motors Corp, 
1987-1989: Production Manager for General Dynamics Corp, 
1980-1987: Eastern Marketing Manager for Ford Motor Company 
1974-1980: Sales trainee, then sales manager, General Specialty Corp. 
EDUCATIONAL HISTORY: 
1970-1974: Harvard Business School, Graduated, M.B.A. 
1966-1970: Brown University, Pre-Med Major 
1962-1966: Benedict Amold High School, New Rochelle, New York, 
Academie Major, Class Valedictorian 
HOBBIES AND INTERESTS: 
Member, Socialist Worker's Party, Students For A Democratic Society. 
Go to the copy machine, run off a hundred or so copies, and 
send one to each company on your list. Be generous and give 
every possibility the benefit of the doubt. Don't bother forging 
your target's name to a covering letter: the résumé will be 
enough. 
Chapter Five 
Provoking Confrontations 
91 
Two weeks later, follow up with a telephone call to each 
company on the list. Giving your target's name, ask why they 
haven't replied to him. Be forceful at first, then become abu- 
sive. Sprinkle the conversation with profanity, especially if 
you're speaking with a woman. Lace your diatribe with phrases 
such as "dumb broad," Even a woman with no interest in femi- 
nism will take offense if you attribute ber alleged stupidity to 
her sex. The more nasty, vulgar, and abusive you are, the more 
vivid an impression you'll make, ensuring that your target's 
name will be remembered if he ever applies to that company for 
employment. 
The Cheating Heart 
If your spouse, “Nancy,” is cheating on you with your best 
friend, this is both bad and good news. The bad news is the 
crushing sense of betrayal that will fall upon you like a ton of 
bricks. The good news is that your easy access to both will 
provide an excellent opportunity to provoke severe anxiety by 
exploiting disinformation. Once you form an idea of how long 
the affair's been going on, you can plan your approach. 
During planning, remain on good terms with both your wife 
and your "best friend." However, the plan has a built-in safety 
factor if you lack the total self-control you need to disguise your 
feelings. In one respect, you'll have to have a lot of emotional 
fortitude, because this devastating plan requires you to tell a 
very damaging lie about yourself, while delivering a devastating 
mind-burn. 
Begin by allowing enough time, at least a year, to pass be- 
tween the beginning of their affair and the moment you act. At 
the appropriate moment, take your “best friend” off to a quiet 
comer:  Gaslighting 
92 
You: J guess you may have noticed something's been on my 
mind during the last few months. 
Best friend: Yeah, I noticed you haven't been yourself lately. 
You seem to have something serious on your mind. 
At this point, your target's guilty conscience may suggest to 
him that you've discovered his betrayal and are about to con- 
front him. This produces enough anxiety so that he'll accept 
your next statement with a profound sense of relief, and 
because you're confessing to him, with total credulity, 
You: J guess you never suspected this, but I'm gay. I don't 
think Nancy knows, either, but since before we were mar- 
ried, I've been seeing this guy, and we've been having sex 
at least once a week. 
Friend: (shocked) No, I never knew. 
You: Well, it's worse than that. My buddy Charlie also goes to 
bars, and he's been picking up guys for ane-night sex. 
About two years ago, I had a routine check-up and my 
blood tested positive for HIV. So Charlie never told me he 
had it. 
Friend: HIV? Have you told Nancy? 
You: No, / didn't have the guts, If 1 told her I'd gotten HIV, I'd 
have had to tell her the whole story, and I couldn't bring 
myself to do that, I know she's got it, because you don't test 
positive until months after you get it, and by that time, 
you've been contagious for a long time. 
If you want to lay it on thickly, you begin crying, repeating 
that you've murdered your wife with the virus. This is maudlin, 
but a very easy story to believe. 
This conversation leaves your best friend wondering about 
his status, and trying to cope with not having sex with your 
wife again to avoid catching AIDS, if he hasn't already been 
Chapter Five 
Provoking Confrontations 
93 
infected. If he's married, he will wonder if he’s spread it to his 
wife, and if his wife's pregnant, the complications become 
mind-boggling and terrifying. 
This isn't a male-only technique. Wives also sometimes 
have to cope with a close friend carrying on an affair with their 
husbands. A similar tactic works against an unfaithful husband 
and the wife's close friend. If your husband's cheating with your 
neighbor and best friend, and she is married, here's one possible 
approach, Next time you and your neighbor are having coffee 
together, the conversation may go like this: 
She: 7 went to the doctor last week, and he called me yesterday 
to tell me I'm pregnant. 
You: That's very good news! I know you and Joe have wanted 
a child for a long time. 
She: Yes, I'm finally pregnant, and I just wonder how long it'll 
be until 1 know jf it's a boy er girl. 
You: / don't think I'll ever get pregnant. We decided not te 
have any children because Harry's got a Down's syndrome 
gene in his make-up. His brother had a boy with Down's 
syndrome, and we don't want to risk going through that. 
At this critical juncture in the conversation, you may notice 
your neighbor and best friend become pale, as she considers the 
implications. When having regular sex with two or more men, 
it's impossible to be absolutely sure who the father is, as even 
the best methods of birth control aren't 100 percent certain. 
She'll blame your husband, her lover, for not telling her of this 
genetic deficiency, although she may not confront him. She'll 
also have to face the difficult decision of whether to have the 
child and risk bringing forth a mental defective, or to have an 
abortion. If she chooses abortion, how will she justify this to 
her husband?  Gaslighting 
94 
Making this work properly requires careful preparation on 
your part, to make it credible. Genetic disorders vary with eth- 
nic background, and you have to survey the latest research to 
find the disorder that fits your husband's ethnicity best. Blacks 
are vulnerable to sickle-cell anemia, while Jews tend to inherit 
Tourette's and Tay-Sachs Syndromes. Know your facts before 
you act. 
The Loan 
Another way to use disinformation to bash your target in the 
chops is to plant faked "news" about a loan application. Obvi- 
ously, you have to be close enough to your target to know the 
details of his plan. 
If your target tells you of his intention to buy a house, you 
can deftly elicit some critical details. You need to know if he 
has to give his landlord written notice of his intention to vacate 
the premises, the name of his mortgage company, and the date 
of the application. 
With this information, you recruit an accomplice whose 
voice your target won't recognize, and who telephones him 
posing as an agent of the mortgage company. He informs him 
that his application's been approved, and that closing can be 
within 30 days. Your target, elated by the news, gives his land- 
lord notice, and when he discovers the hoax, will find himself 
out on the street if his apartment has already been rented. 
You can do the same with an auto loan, although this won't 
cause your target as much dislocation as the fake home loan 
approval. Although many banks and auto dealers today have 
instant approval for loan applications, your target may apply to 
a company that takes longer. If so, he's wide open to a disin- 
formation scam. If you know that your target plans to sell his 
present vehicle privately, a fake call telling him his auto loan 
Chapter Five 
Provoking Confrontations 
95 
has been approved can move him to advertise it for sale. At the 
very least, he'll waste time showing it to prospective buyers 
before he's prepared to sell it, assuming his loan application 
eventually goes through. If he accepts a deposit, he'll be obliged 
to sell it, however inconvenient it may be, 
The Ladies’ Man 
If your target is a ladies' man, follow him when he goes out 
on a date. If he takes his date into a restaurant, theater, or any 
other establishment where you can have him paged, have a fe- 
male accomplice pose as his "wife" and telephone the man- 
ager’s office and request the manager to page your target, citing 
an emergency at home to give it urgency. The beauty of this 
tactic is that it works whether your target is actually married or 
not, His date will be shocked to discover that he has a wife 
somewhere. The other nice aspect of this is that it works against 
female targets as well. 
One possible problem with this tactic is that paging can be 
very discreet, with a simple announcement stating: "Will Mr. 
Harry Target please come to the manager's office?" This doesn't 
disclose the reason, and can fall flat. 
Throwing Gasoline on the Flames 
Creating a conflict between your target and another person 
takes a fine hand, but it can often be even more productive to 
exploit and aggravate an existing one, when a little effort can 
go a long way. The basic tactic is to do something to further 
antagonize one of the parties, in a way that although your target 
knows he didn't do it, he won't be believed if he denies it. Ob- 
viously, timing is everything. Let's examine several scenarios to 
see the opportunities and to lay out tactics and timing appro- 
priate to various situations:  Gaslighting 
96 
Your target has become involved in an argument 
with his boss on a controversial social or political ts- 
sue, such as abortion. The supervisor is pro-abortion, 
and your target is a right-to-life advocate. Ever alert to 
the ramifications, you place a bumper sticker on the 
supervisor's car, The sticker reads; "Abortion is Mur- 
der." Guess who the principal suspect will be when the 
boss sees what's on his bumper? 
The bumper sticker tactic is for when you don't want to do 
any physical damage and the owner of the vehicle can take ef- 
fective reprisals, You can also use the bumper sticker tactic as a 
follow-up to a situation you've created, such as parking your 
target's car in the supervisor's parking place. If you can't obtain 
the proper bumper sticker quickly enough, other tactics are 
possible: 
After your target has had a confrontation with his 
boss, you go to a pay phone late at night and ring the 
supervisor's number. When he answers, say nothing 
and let him hang up, then call him again in ten minutes. 
Keep this up most of the night, changing locations after 
the second call and each subsequent one to negate 
tracing by the telephone company. You'll immediately 
know if your tactic is successful if you hear the person 
you're calling begin cursing out your target and threat- 
ening to settle accounts at the office the following day, 
In other situations, you have to be more forceful: 
Your target, a salesman, is involved in an intense ri- 
valry with another salesman, and has just accused the 
other of raiding his territory and pirating some of his 
customers. You have reason to dislike both, and decide 
to act quickly before the incident blows over, by slash- 
Chapter Five 
Provoking Confrontations 
ing the tires of the other salesman. Obviously, your tar- 
get is the prime suspect, leading ta more trouble. If you 
want to take this a step further, the next night you siash 
your target's tires. Meanwhile, you have a female ac- 
complice telephone your target's clients for appoint- 
ments, posing as the rival's secretary, The news will 
quickly get back to your target. 
This can easily lead to a fist-fight, or worse, so you have to 
be sure you want to involve the other party to this extent, and 
not bring harm to an innocent person. 97 
Similar tactics work when there's a neighborhood dispute. 
You overhear your target telling of a dispute with 
his neighbor, who repeatedly parks his car with one set 
of wheels on the target's lawn, Late that night, you go 
to your target's house, and if the offending car's parked 
half-way on his lawn, you flatten the tires, The neigh- 
bor will blame only one person, and it won't be you! 
Exploit other opportunities as well: 
Your target lives in an apartment, with a neighbor 
who plays his stereo or TV too loudly, and repeated 
complaints have produced no satisfaction. That 
evening, you go to the location, and if you hear the 
sound turned up loudly, go to a pay phone and call the 
neighbor. Scream "Turn it down" at him, so loudly that 
he can't be sure the voice is not the target's voice. Then 
go back to the apartment and throw a brick through the 
neighbor's window. Of course, the neighbor will storm 
out of his apartment and begin banging on your tar- 
get's door, and the fun will begin. As usual, you must be prepared to take immediate advantage of 
every fleeting opportunity:  Gaslighting 
98 
There are several possible variants on this scenario. One is 
based on the concern that you might be seen running away, or 
that the neighbor might pursue and catch you. The way to cope 
with this is to telephone the neighbor during the small hours of 
the morning, screaming, "Now I'll keep you awake!" and throw 
the brick through his window once you see the lights go out 
again. Still another way is to catch the neighbor asleep when 
you throw the brick, to be sure of getting away before he be- 
comes fully awake and gets on any clothing. You can telephone 
him afterward, and threaten that the next time you'll use a bottle 
of gasoline. When police arrive, you'll have been long gone. 
If the dispute involves an animal in any way, you can take 
advantage of the road-kill we so often see on our streets; 
Your target's dog has bitten a neighbor's child. This 
is serious enough as it stands, but you want to add to 
the problem. Late one night, throw a dead dog onto 
your target's door-step to be found in the morning. This 
silent act is sure to be perceived as a threat by your 
target. 
/ With enough such episodes, neighbors may begin to per- 
ceive your target as a crank. You can, of course, help the proc- 
ess gather momentum. 
The Neighborhood Crank 
Once your target has begun to lose credibility with his 
neighbors, a few more efforts will build a reputation as the 
neighborhood crank. The basic tactic is a series of unfounded 
complaints. Note that it will be very helpful if your voice 
sounds like that of your target. If not, make a serious effort to 
find an accomplice who can mimic your target's tone and 
accent. The results will be devastating. 
Chapter Five 
Provoking Confrontations 
99 
If your target has his newspaper delivered, call the newspa- 
per's circulation office a couple of times per week, giving his 
name, and complaining that his newspaper never arrived that 
morning. It helps to be abusive. If the route manager arrives 
with another copy to replace the missing one, and sees it at his 
door, it won't take long for your target to become known as a 
nut case at the newspaper office. 
You can make this work even if your target is an early riser 
and takes in his paper as soon as it arrives. Simply telephone 
the circulation office before the time you know he awakens, and 
complain that the paper isn't there yet. It's almost certain that 
the staffer who answers the telephone will reply that they don't 
consider a delivery missed until 6 or 7 A.M., but you remain 
completely unreasonable, and insist on getting your paper 
RIGHT NOW! 
If you or your accomplice sound like your target, telephone 
one of his neighbors late at night and complain about the loud 
music, If you refrain from making the call until you've seen the 
neighbor's lights go out, the person will be peeved at having 
been awakened to hear a spurious accusation. 
The next tactic works only if the local police do not have a 
911 system, similar to “Caller 1.D.," that displays the caller's 
number on the operator's screen. From a pay telephone, call the 
police to complain about your target's neighbor's loud music. A 
few such unfounded calls and the police will consider him a 
crank too. This can have very beneficial side-effects by destroy- 
ing his credibility if he ever calls the police to complain that 
someone is doing things to him. 
If the police have a caller display system, you can still make 
this tactic work if you have a spare handset and the skill to tap 
into your target's line. Thus, any call you make this way will 
display as coming from your target's number, and you can make 
crank calls to the police almost at will. The only restriction youGaslighting 
100 
must observe is to do it when you know your target's not using 
his telephone, so that he won't hear the extra voice on his line. 
Notes 
1. Computer Viruses, Worms, Data Diddlers, Killer Pro- 
grams, and Other Threats to Your System, John McAffee 
and Colin Haynes, NY, St. Martin's Press, 1989, pp. 89-91. 
Chapter Six 
Real Trouble 
101 
\chapter{Real Trouble}
Chapter Six 
REAL TROUBLE 
There are several ways to place bureaucratic booby-traps 
designed to cause your target serious trouble, legal and other- 
wise, in his immediate environment. For these tactics, it helps 
greatly to have access to your target's wallet, home, vehicle, and 
mail. 
Missing Papers 
State law usually requires a motor vehicle operator to have 
with him not only his driver's license, but the vehicle registra- 
tion and, in states with mandatory liability insurance, the vehi- 
cle's insurance certificate or card. If ever your target leaves his 
vehicle unlocked, or if you have access to it another way, such 
as "borrowing" his keys, just remove the registration form and 
insurance card from the glove compartment. People don't nor- 
mally check to verify that their registration and insurance 
paperwork are in the vehicle each time they get in the car, and 
the next time your target gets into an accident, or is stopped by 
a police officer, he'll get a citation for not carrying the required 
paperwork with him in the vehicle. 
Gaslighting 
102 
The Fake Divers License 
Access to your target's wallet allows the simplest of these 
tactics, Simply borrow his drivers license long enough to make 
a same size Polaroid photo of it, and retum the fake to his wal- 
let. 
The Polaroid Corporation makes several same-size copying 
attachments, the simplest of which is a close-up lens. This is 
good enough to produce a copy that will appear real to casual 
inspection, such as when your target flips through the credit 
card envelopes in his wallet. However, if he has to use the card 
as credentials, such as when making 2 purchase with a check, 
or show it to a police officer, the fake will stand out like a sore 
thumb. 
The reason? State drivers licenses are designed to resist for- 
gery. Some have a fine network of lines printed through the 
photo, and others, such as New Mexico's drivers licenses, have 
a stamped lettering of the state's name that appears when hold- 
ing the card at a certain angle. Yet others have a logo that does 
not reproduce when photographed. Presenting a forged drivers 
license to a police officer will produce serious repercussions, 
even if the motor vehicle bureau's computer verifies that your 
target has a valid license, 
The Fake Registration 
Making a vehicle registration slip that is an obvious copy or 
forgery takes even less equipment. Once you borrow the real 
registration from your target's vehicle, make a photocopy on an 
ordinary copy machine, and return the copy to the vehicle. A 
police officer will be able to spot the copy, because it will lack 
the colored validation stamp, which reproduces in black on a 
copy machine, and your target will have some explaining to do. 
Chapter Six 
Real Trouble 
103 
You can guarantee results by a simple extra step. Using 
white-out, obliterate one of the license plate numbers on the 
registration before making the copy, then slip the copy into 
your typewriter and type in another number or letter. A police 
officer will immediately notice that the registration and plate do 
not match, a fact which will be verified when he calls in for a 
check with the state motor vehicle bureau computer. Then the 
ship will hit the sand. 
An even better way is to intercept your target's application 
for a new registration, and mail back to him a fake of your own 
making. The plate numbers match, and even the envelope is 
real, as you've saved the one in which your registration arrived. 
You won't be able to supply a validation sticker for his plate, 
but your target may think this was merely an oversight by a 
clerk in the motor vehicle bureau. If he doesn't pursue this, 
feeling secure because he has a valid (he thinks!) registration 
certificate in his glove compartment in case he's stopped, he'll 
find out differently the first time a police officer stops him. 
The vehicle registration will have lapsed, and he'll become 
aware of it when the officer ominously orders him out of the 
car, possibly even at gun-point, If the officer stopped him for 
drunk driving, his fate is sealed. He's going to jail, because the 
combination of DUI and a fake registration will leave the offi- 
cer no choice. 
One possible problem is that your target may notice that his 
check to the motor vehicle bureau never cleared his account. 
This omission, while it would alert a normally functioning hu- 
man being that something, somewhere, has gone wrong, may 
pass completely unnoticed, because by the time you get to this 
stage, your target will already be suffering mind-damage.  Gaslighting 
104 
The Stolen Car 
One way to guarantee that your target's vehicle will come to 
the attention of the police is to report it stolen. You don't need 
access to your target's telephone for this. Simply call from a pay 
phone at a shopping mall, and state that your car disappeared 
while you were inside the mall. 
Fake Hunting License 
You can perform the sane sort of forgery with a hunting li- 
cense. Intercepting your target's application and substituting a 
forgery by return mail will cause him a pack of trouble when a 
game warden asks to see his license. 
Forgery Tips: 
Making Real looking Fakes 
With official documents carrying several security features to 
frustrate forgery, you have to be able to produce a fake that 
looks good enough for casual inspection, but will surely show 
up as a fake in an official inspection. Polaroid photography is 
good enough to produce an acceptable drivers license, so that 
your target won't notice it’s not real, 
If the forgery is too crude, even your target, dull, brutish 
and insensitive as he may be, will notice something wrong with 
it. A little care goes a long way. What you need is a forged 
document that won't scream "fake," so that your target's hunting 
buddies don't begin wondering why his hunting tag is printed 
only in black. 
First, try to duplicate the real document's color, texture, and 
stiffness as closely as you can, If the drivers license is lami- 
nated, pass your Polaroid forgery through a plastic laminator. 
Chapter Six 
Real Trouble 
105 
If you're faking a vehicle registration printed on ordinary 
bond paper, use similar paper to make your copy. A color stamp 
can cause a special problem, because color is one of the most 
conspicuous attributes on the face of a document, and a black- 
only imitation will stand out too much to pass unnoticed. 
Some copy machines, such as the Canon PC6-RE and sey- 
eral Minolta models, use toner cartridges, and the manufacturer 
sells cartridges with color toners for those who want to copy in 
color. This copier's instruction manual even describes how to 
produce two-color printing on one sheet of paper. This merely 
requires removing the colored part of the document from the 
black original to avoid printing it in black. Making the colored 
imuge requires pasting the colored stamp in the same location 
on a white sheet of paper, changing the toner cartridge, and 
running the copy through a second time. Two passes, and 
you've got your realistic forgery. 
Your target's problem will occur because the numbers 
printed on the forgery are fake, and will not pass inspection. 
They won't match his vehicle's plate, or VIN, and the officer 
will become very inquisitive. 
You can help this along while making your Polaroid photo- 
graph of his drivers license by altering his registration or li- 
cense number. Type a few fake digits on an oblong piece of 
white paper and lay it over the real number while 
photographing it. A police officer scrutinizing the license will 
immediately pick up on the disparity, and a radio check will 
confirm it. 
An important point regarding detecting forgery is mind-set. 
Police officers are aware that some people travel on falsified 
papers, and they routinely double-check with the computer to 
verify the status of the document. Your target, on the other 
hand, won't be expecting a fake. After all, he just sent in his 
money to pay for a license, registration or insurance card, and  Gaslighting 
106 
he's legally entitled to one. The one he receives in the mail has 
to be real, because a state agency sent it. This mental factor 
works for you whenever you substitute forgeries in his wallet or 
glove compartment. 
Plate- Swapping 
A quick and dirty tactic, if your target has two vehicles and 
you cannot gain access to his keys, is to swap the registration 
plates. Most people don't check their license plate numbers 
each day, and as long as they see a plate where it's supposed to 
be, that's enough for them. However, it won't be enough for the 
first police officer to stop either vehicle, because the plate and 
registration won't match. 
Don't swap plates with a vehicle belonging to an innocent 
person. You don't want to cause anyone uninvolved with your 
target problems. 
Bad Checks 
One way to cause mind-bending and aggravating problems, 
including criminal charges, for your target is to issue bad 
checks in his name. This does not involve forging checks from 
scratch, but signing his name to genuine checks drawn on an 
account with insufficient funds. 
The first step is to obtain his account number from dis- 
carded deposit slips, canceled checks, or by simply copying it 
out of his check-book. Practice signing his name, until you can 
produce a reasonable facsimile. Next, you go to another branch 
of his bank far from his home or workplace, where it's unlikely 
that he’s known by sight. You open a second account in his 
name, depositing a minimal amount of money, Use a mail drop 
to receive the printed checks.! When they arrive, you're ready to 
go to work. 
Chapter Six 
Real Trouble 
107 
If you have access to his incoming or outgoing mail, pay 
some of his bills using the new checking account. If you can get 
hold of his outgoing mail, open the envelope flaps with a wet 
sponge and remove his check, substituting your own. The 
checks will bounce, of course, and your target will find his 
utilities cut off, and perhaps his car repossessed and his bank 
accounts taken over by the Internal Revenue Service. 
Luck can play a part, as well. If you're lucky enough to find 
a starter check-book in a dumpster or elsewhere, use those for 
paying your target's bills. As starter checks are for temporary 
use, only until the printed checks arrive, many people discard a 
partly-full book upon receipt of the printed checks. This saves 
you the trouble of impersonating your target. 
Eventually, the truth will come out in the wash, but the time 
this takes appears infinite to the person being pursued by 
creditors, who simply don't want to hear any stories or any 
statements except "Here's your money.” 
Contraband 
Remember the illegal drug and the handgun you obtained a 
long time ago, for use in a certain contingency? Well, here are 
several ways you can use them to put your target in very deep 
doo-doo. 
If your target's on his way to the airport to catch a flight, 
and you get the opportunity to slip your untraceable handgun 
into his briefcase, the airport security system will do the rest. 
Once his handgun shows up on the X-ray screen, he'll have so 
much explaining to do that he'll probably miss his flight, even if 
he doesn't suffer immediate arrest. 
If you know that there are drug-sniffing dogs in use at this 
airport, sprinkle the small amount of illegal drug you've saved 
in his pants cuff or jacket pocket. He will find himself detained  Gaslighting 
108 
and searched by airport police after the drug dog “alerts” to the 
odor of the drug on his person. 
If your target is going to visit a relative or friend in state 
prison, this provides an opportunity to employ any of several 
tactics. The typical state or federal penitentiary has one or more 
signs at each approach, which read something like this: 
STOP! 
WARNING! 
You are entering prison grounds. It is absolutely 
forbidden to bring any firearms, ammunition, alcoholic 
beverages, illegal drugs, or other contraband onto 
prison grounds. If you have any such materials in your 
vehicle or on your person, do not proceed beyond this 
sign, All vehicles and all visitors are subject to search, 
and anyone found in possession of contraband will be 
arrested. 
Place a handgun under one of the seats in your target's ye- 
hicle. If you weren't able to obtain an untraceable handgun, use 
the illegal drug. Lacking even this, put a bottle of booze under 
his seat or in his glove compartment. You don't even have to 
spend much money on the liquor. A small airline-size bottle 
will appear very incriminating, because its small size makes it 
very concealable, ideal for smuggling into a prison. 
To make certain that your target and his vehicle will receive 
the scrutiny they deserve, make an anonymous telephone call 
from a pay phone. Describe your target and his vehicle, even 
providing the license plate, and tell the prison administration 
that Harry plans to break out his brother Bill that afternoon. 
The authorities will do the rest, 
Chapter Six 
Real Trouble 
109 
Snitching Him Off to the Cops 
One way to cause real trouble for your target is to inform 
the authorities of any illegal activity in which he's involved. In 
most cases, it's perfectly possible to inform anonymously, as 
police agencies operate "hot lines" for anonymous tippers. In 
some cases, especially those involving illegal drugs, there's 
even a reward for information. 
However, it's possible to snitch him off even if he's walking 
the straight and narrow. An unfortunate fact of American life in 
the Twentieth Century is that some agencies operate as loose 
cannons, conducting arrests and raids on meager evidence, or 
even no evidence at all, The instance of the Bureau of Alcohol, 
Tobacco, and Firearms in Waco, Texas, is too well-known to 
warrant discussion here, The main point, however, is that the 
Waco incident wasn't unique: it was only the best-publicized 
because of its extremely violent nature and large casualty count. 
BATF Agents raided the home of a Toledo, Ohio, man after 
receiving an anonymous letter that he was involved in a "White 
supremacist plot.” The "plot" allegedly was to bomb a predomi- 
nantly Black public housing project, but federal officers admit- 
ted in court that they had no evidence to support this allegation. 
Agents did not identify themselves as law officers when they 
broke in, and the man opened fire at them, but was acquitted of 
the charge of "aggravated menacing" by a jury.? 
This instance wasn't unique. U.S. Drug Enforcement 
Agency (DEA) officers have raided the wrong premises, and 
some local agencies have even planted narcotics to be 
"discovered" during a raid. The fact is that some law enforce- 
ment officers, and the agencies employing them, are "hot dogs" 
who will do anything to make a bust, and will make an arrest or 
apply for a search warrant on little or no solid evidence. This  Gaslighting 
110 
provides an opening for you if you decide that your target de- 
serves a Visit from the cops. 
An anonymous letter is the best way to go, When compos- 
ing your letter, be as specific as you can, even if you have to 
fabricate facts or slant them to suit your case. If, for example, 
you know that your target goes to the movie theater every Tues- 
day night, state that he meets his drug wholesaler at the theater. 
Officers surveilling him will note that he does, as stated in your 
letter, go to the theater each Tuesday night, and they'll draw the 
conclusion you want from this confirmation of your allegation. 
Protecting Yoursell 
Obviously, forging official paperwork, stealing your target's 
mail, and writing checks in his name can cause serious reper- 
cussions for you if you're caught, You may end up in worse 
trouble than he, and you'll wish you had never begun. 
This is why it’s better to be safe than sorry. If you have any 
doubts about your security, abandon that phase of your pian 
and concentrate on the parts that don't expose you to risk. 
Regarding risks, let's examine them, and note a few ways to 
counter or minimize them. In carrying out some of the more 
aggressive tactics, such as opening a checking account in your 
target's name and purloining his mail, you run the risk of being 
identified and of leaving fingerprints. There are ways of reduc- 
ing these risks almost to zero. 
Let's tackle the easy one first. Always wear plastic gloves 
when handling anything that you're going to send in your tar- 
get's name. If you forge a letter, write a check, or redirect mail, 
never handle these items bare-handed. 
Be especially careful when handling mail. Remember that 
the U.S. Postal Inspection Service is a low-profile but extremely 
proficient law enforcement agency, better than their peers. Over 
Chapter Six 
Real Trouble 
11 
the years, their conviction rate has been higher than the other 
two well-known federal agencies, the FBI and the Secret 
Service. You surely don't want the Postal Inspectors on your 
case. 
Identification is another problem. Obviously, don't be seen 
lifting envelopes from his desk or entering his home. Reducing 
the risk while opening a checking account is more difficult be- 
cause you have to show your face, and you may even have to 
show some identification, 
There are two ways to ensure that a bank clerk won't be able 
to identify you. One obvious way is to wear a disguise, This 
doesn't mean a total cosmetic make-over, but a simple disguise, 
designed to obscure part of your face. If you're normally clean- 
shaven, wear a false mustache. If you don't wear glasses, put on 
a pair for the occasion. Of course, always wear a hat, to obscure 
your hair color and style, but don't wear the type of hat you 
normally wear. If you like baseball caps, wear a captain's cap or 
a fedora, Better yet, if you've bought a hat of the sort your tar- 
get wears for the size-switching tactic, wear that, if it fits your 
head, to complete the disguise. 
The other aspect of impeding recognition is time. Open the 
fake checking account long before you plan to use it, as a wil- 
ness's memory fades with time. Months or a year later, the bank 
teller who served you won't even remember the incident. 
Finally, you may have to show identification. This isn't 
likely if you already have a deposit slip or can provide the 
number of your target's current account at the bank. Bank per- 
sonnel pay closest attention to people trying to withdraw 
money, not deposit it, and this reduces the scrutiny you'll re- 
ceive, 
If push comes to shove, however, and the teller insists on 
seeing identification, reach for your wallet, and exclaim with  Guslighting 
112 
surprise that you must have left it at home. Excuse yourself 
with a promise to return the next day, and vanish forever. 
Notes: 
1. How to Use Mail Drops fer Privacy and Profit, Jack Luger, 
Loompanics Unlimited, Port Townsend, WA, 1988. 
2. Gun Week, October 22, 1993, p. 2. 
Chapter Seven 
A Final Word 
113 
\chapter{A Final Word}
Chapter Seven 
A FINAL WORD 
Now that you've examined the variety of gaslighting tech- 
niques ayailable, you can choose those that will work best 
against your target while exposing you to the least risk. Always 
remember that not all techniques work in all situations. The 
name of the game is "survival," so don't take unnecessary 
chances. Choose carefully, to allow you to enjoy your target's 
discomfort and deterioration. 
Seeking A Final Solution 
In the beginning, gaslighting proceeds with a light touch, 
allowing you to take it as far as you wish. Later, many direct 
gaslighting tactics are forceful and discrete, for use when your 
target's on the way down. They are the equivalent of sinking a 
ship and machine-gunning survivors in their life-boat. Which 
you choose to employ depends on what you feel your target de- 
serves.

\end{document}